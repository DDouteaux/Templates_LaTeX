\documentclass[11pt,a4paper,rgb]{report}

%% Importation de tous les fichiers de configuration nécessaires.
%%---------------------------------------------------------------%%
%                                                                 %
%                        PACKAGE IMPORTS                          %
%                        Module : Common                          %
%                                                                 %
%        All packages should be compatible with XeLaTeX           %
%                                                                 %
%%---------------------------------------------------------------%%


% Global libraries
\usepackage[textwidth=18cm,bottom=2cm,top=2cm]{geometry} % pages borders
%\usepackage[top=2cm, bottom=2cm, outer=6cm, inner=1.5cm, marginparwidth=5cm, marginparsep=.75cm]{geometry}
\usepackage[french]{babel} % language type annotations

% Graphics
\usepackage{graphicx} % improve include graphics
\usepackage{tikz} % for drawing
\usepackage[most, breakable]{tcolorbox} % to have fill paterns for TikZ
\usepackage{tikzscale} % in order to scale TikZ pictures

% Tabular
\usepackage{booktabs} % to have predefined tab rules
\usepackage{array} % to define new column types
\usepackage{longtable} % in order to have multipage tables

% Colors
%\usepackage{color}
\usepackage{xcolor} % to have access to every immaginable color

% TikZ libraries
\usetikzlibrary{fit} 
\usetikzlibrary{graphs} % package for graphics representation
\usetikzlibrary{positioning} % possitionning inside TikZ picture
\usetikzlibrary{arrows} % beautiful arrows with TikZ
\usetikzlibrary{calc} % enable computation of values insidecture
\usetikzlibrary{decorations} % decoration for TikZ picture
\usetikzlibrary{matrix} % matrix of nodes
\usetikzlibrary{angles} % angle helpers

% Characters
\usepackage{pifont} % in order to have checkmarks and crosses
\usepackage{eurosym} % to have euro symbol

% TOC and titles
\usepackage{caption} % to redefine caption styles
\usepackage[explicit,pagestyles]{titlesec} % redefine titles style
\usepackage{titletoc} % for partial TOC in chapters

% LaTeXing
\usepackage{etoolbox} % to have ifstrempty and other command in macros
\usepackage[unicode, hidelinks]{hyperref} % math formulas in section names

% Headers and footers
\usepackage{fancyhdr} % for fancy headers and footers

% Multicolumn
\usepackage{multicol} % provide multicolumn environments

% Index
\usepackage{makeidx} % enable the computation of document index

% Mathematics
\usepackage[d]{esvect} % Fancy vector arrows
\usepackage{tkz-tab} % For variation arrays

% Itemize
\usepackage{enumitem} % Allow to set values and positions for items

% Testing
\usepackage{lipsum} % for testing
\tcbuselibrary{listings,skins}

% Fonts
%============ KURIER ==========================
%\usepackage[math]{kurier} % main font
%\usepackage[T1]{fontenc} % font encoding
%\DeclareTextCommandDefault{\nobreakspace}{\leavevmode\nobreak\ } % Because T1 disable this command, we redefine it
%============ EULER ===========================
%\usepackage[T1]{fontenc}
%\usepackage{mathpazo} % add possibly `sc` and `osf` options
%\usepackage[T1]{eulervm}
%============ MATH DESIGN =====================
\usepackage{fontawesome} % to have web fonts
\usepackage[bitstream-charter]{mathdesign} % nice math and common fonts






\iffalse
%% Libraries for graphics and colours
\usepackage{changepage} % for adjustwidth command
\usepackage{colortbl}

\usepackage{mdframed} % permet de couper du texte entre plusieurs pages
\usepackage{framed}


\usepackage{stmaryrd}
\usepackage{ifthen} % use to make if then conditions


\usepackage{capt-of}
\usepackage{xspace}

\usepackage[normalem]{ulem}
\usepackage{url}
\usepackage{subcaption}
\usepackage{remreset}
\usepackage[framemethod=tikz]{mdframed}

\usepackage{listings}

%% Other libraries
\usepackage{notoccite} % helps for quotations
\usepackage{float} % helps to place figures
\usepackage{setspace} % in order to help adding space in the document
\usepackage{multirow} % in order to use multirow in a table

\usepackage{rotating} % to have rotating boxes
\usepackage[makeroom]{cancel} % cancel symbols in maths
\usepackage{marvosym}
\usepackage{xfrac}
\fi
%%-------------------------------------------------------------------%%
%                                                                     %
%             PREDEFINED COLORS AND COLOR RELATED COMMANDS            %
%                                                                     %
%%-------------------------------------------------------------------%%


%%==================================================================
%% FIRST PART : PREDEFINED COLORS
%%==================================================================

%%
%% Various colors
%%
\definecolor{vertforet}{RGB}{0,153,51}
\definecolor{bordeau}{RGB}{199,16,26}
\definecolor{gris}{RGB}{195,195,195}
\definecolor{bluenight}{RGB}{0,82,148}
\definecolor{violet}{RGB}{112,4,98}
\definecolor{amber}{RGB}{255,110,0}


%%
%% Tints of green
%%
\definecolor{green1}{RGB}{0,100,0}    % vert forêt 
\definecolor{green3}{RGB}{85,107,47}
\definecolor{green4}{RGB}{0,128,0}


%%
%% Tints of red
%%
\definecolor{red1}{hsb}{1,1,1}      % bordeau
\definecolor{red2}{hsb}{1,.9,1}
\definecolor{red3}{hsb}{1,.8,1}
\definecolor{red4}{hsb}{1,.7,1}
\definecolor{red5}{hsb}{1,.6,1}
\definecolor{red6}{hsb}{1,.5,1}
\definecolor{red7}{hsb}{1,.4,1}
\definecolor{red8}{hsb}{1,.3,1}
\definecolor{red9}{hsb}{1,.2,1}
\definecolor{red10}{hsb}{1,.1,1}
\definecolor{red11}{hsb}{1,0,1}


%%
%% Tints of blue
%%
\definecolor{blue1}{hsb}{0.594,1,1}
\definecolor{blue2}{hsb}{0.594,.9,1}
\definecolor{blue3}{hsb}{0.594,.8,1}
\definecolor{blue4}{hsb}{0.594,.7,1}
\definecolor{blue5}{hsb}{0.594,.6,1}
\definecolor{blue6}{hsb}{0.594,.5,1}
\definecolor{blue7}{hsb}{0.594,.4,1}
\definecolor{blue8}{hsb}{0.594,.3,1}
\definecolor{blue9}{hsb}{0.594,.2,1}
\definecolor{blue10}{hsb}{0.594,.1,1}
\definecolor{blue11}{hsb}{0.594,0,1}


%%
%% Tints of violet
%%
\definecolor{violet1}{hsb}{.75,1,1}    % violet
\definecolor{violet2}{hsb}{.75,.9,1}
\definecolor{violet3}{hsb}{.75,.8,1}
\definecolor{violet4}{hsb}{.75,.7,1}
\definecolor{violet5}{hsb}{.75,.6,1}
\definecolor{violet6}{hsb}{.75,.5,1}
\definecolor{violet7}{hsb}{.75,.4,1}
\definecolor{violet8}{hsb}{.75,.3,1}
\definecolor{violet9}{hsb}{.75,.2,1}
\definecolor{violet10}{hsb}{.75,.1,1}
\definecolor{violet11}{hsb}{.75,0,1}


%%
%% Tints of orange
%%
\definecolor{orange1}{hsb}{.1,1,1}   % ambre
\definecolor{orange2}{hsb}{.1,.9,1}
\definecolor{orange3}{hsb}{.1,.8,1}
\definecolor{orange4}{hsb}{.1,.7,1}
\definecolor{orange5}{hsb}{.1,.6,1}
\definecolor{orange6}{hsb}{.1,.5,1}
\definecolor{orange7}{hsb}{.1,.4,1}
\definecolor{orange8}{hsb}{.1,.3,1}
\definecolor{orange9}{hsb}{.1,.2,1}
\definecolor{orange10}{hsb}{.1,.1,1}
\definecolor{orange11}{hsb}{.1,0,1}


%%
%% Tints of pink
%%
\definecolor{pink1}{hsb}{.917,1,1}     % rose
\definecolor{pink2}{hsb}{.917,.9,1}
\definecolor{pink3}{hsb}{.917,.8,1}
\definecolor{pink4}{hsb}{.917,.7,1}
\definecolor{pink5}{hsb}{.917,.6,1}
\definecolor{pink6}{hsb}{.917,.5,1}
\definecolor{pink7}{hsb}{.917,.4,1}
\definecolor{pink8}{hsb}{.917,.3,1}
\definecolor{pink9}{hsb}{.917,.2,1}
\definecolor{pink10}{hsb}{.917,.1,1}
\definecolor{pink11}{hsb}{.917,0,1}


%%
%% Tints of yellow
%%
\definecolor{yellow1}{hsb}{.14,1,1}   % jaune
\definecolor{yellow2}{hsb}{.14,.9,1}
\definecolor{yellow3}{hsb}{.14,.8,1}
\definecolor{yellow4}{hsb}{.14,.7,1}
\definecolor{yellow5}{hsb}{.14,.6,1}
\definecolor{yellow6}{hsb}{.14,.5,1}
\definecolor{yellow7}{hsb}{.14,.4,1}
\definecolor{yellow8}{hsb}{.14,.3,1}
\definecolor{yellow9}{hsb}{.14,.2,1}
\definecolor{yellow10}{hsb}{.14,.1,1}
\definecolor{yellow11}{hsb}{.14,0,1}

%%
%% Tints of brown
%%
\definecolor{brown1}{RGB}{210,105,30}    % brown


%%
%% Tints of gray
%%
\definecolor{gray1}{hsb}{0,0,0}
\definecolor{gray2}{hsb}{0,0,.1}
\definecolor{gray3}{hsb}{0,0,.2}
\definecolor{gray4}{hsb}{0,0,.3}
\definecolor{gray5}{hsb}{0,0,.4}
\definecolor{gray6}{hsb}{0,0,.5}
\definecolor{gray7}{hsb}{0,0,.6}
\definecolor{gray8}{hsb}{0,0,.7}
\definecolor{gray9}{hsb}{0,0,.8}
\definecolor{gray10}{hsb}{0,0,.9}
\definecolor{gray11}{hsb}{0,0,1}




%%==================================================================
%% SECOND PART : COMMANDS TO AUTOMATICALLY PLACE COLORS
%%==================================================================

\newcommand\bColor[1]{\textcolor{bluenight}{#1}}
\newcommand\vColor[1]{\textcolor{vertforet}{#1}}
\newcommand\viColor[1]{\textcolor{violet}{#1}}
\newcommand\aColor[1]{\textcolor{amber}{#1}}
\newcommand\gColor[1]{\textcolor{gray}{#1}}
\newcommand\rColor[1]{\textcolor{bordeau}{#1}}
\newcommand\oColor[1]{\textcolor{orange}{#1}}
\newcommand\wiColor[1]{\textcolor{white}{#1}}
\newcommand\wColor[1]{\textcolor{white}{#1}}




%%==================================================================
%% THIRD PART : GLOBAL COLORS
%%==================================================================

% Apply user colors
%
% Parameters :
% - Text to display in given color
\newcommand\curColor[1]{\textcolor{\currentColor}{#1}}
\newcommand\headColor[1]{\textcolor{\headersColor}{#1}}
\newcommand\thColor[1]{\textcolor{themeColor}{#1}}



%%==================================================================
%% FOURTH PART : MISCELLANEOUS
%%==================================================================

% Draw a small square to display a sample of a given color.
%
% Parameters :
% - The color to display
\newcommand{\testColors}[1]{\tikz{ \draw [color=black, fill=#1] (0,0) rectangle (10pt, 10pt); } #1\newline}
%%-------------------------------------------------------------------%%
%                                                                     %
%                           QUICK TEXT STYLES                         %
%                                                                     %
%%-------------------------------------------------------------------%%

%% LIST OF AVAILABLE TEXT STYLES 
%%    - textbi : bold and italic
%%    - textbic : bold, italic and color
%%    - textbith : bold, italic and theme color
%%    - textbth : bold and theme color
%%    - textith : italic and theme color
%%    - textit : italic
%%    - textth : theme color
%%    - textbf : bold
%%    - textsc : small capital
%%    - textscth : small capital theme color
%%    - basic : nothing


% Text bold and italic simultaneously
%
% Parameters :
% - Text to which you want to apply the style
\newcommand\textbi[1]{\textbf{\textit{#1}}}


% Text bold, italic and with given color simultaneously
%
% Parameters :
% - Text to which you want to apply the style
% - The color to apply
\newcommand\textbic[2]{\textcolor{#2}{\textbf{\textit{#1}}}}


% Text bold, italic and in theme color simultaneously
%
% Parameters :
% - Text to which you want to apply the style
\newcommand\textbith[1]{\thColor{\textbf{\textit{#1}}}}


% Text bold and in theme color simultaneously
%
% Parameters :
% - Text to which you want to apply the style
\newcommand\textbth[1]{\thColor{\textbf{#1}}}


% Text italic and in theme color simultaneously
%
% Parameters :
% - Text to which you want to apply the style
\newcommand\textith[1]{\thColor{\textit{#1}}}


% Text sc and in theme color simultaneously
%
% Parameters :
% - Text to which you want to apply the style
\newcommand\textscth[1]{\thColor{\textsc{#1}}}


% Apply a text style format by its name as given in the configuration.
% 
% Parameters :
% - Name of the style
% - Content to which you want to apply the style
\makeatletter
\newcommand{\applyTextStyle}[2]{%
  \ifnum\pdf@strcmp{#1}{textbi}=0%
  \textbi{#2}%
  \else\ifnum\pdf@strcmp{#1}{textbic}=0%
  \textbic{#2}{\currentColor}%
  \else\ifnum\pdf@strcmp{#1}{textbith}=0%
  \textbith{#2}%
  \else\ifnum\pdf@strcmp{#1}{textbth}=0%
  \textbth{#2}%
  \else\ifnum\pdf@strcmp{#1}{textith}=0%
  \textith{#2}%
  \else\ifnum\pdf@strcmp{#1}{textbf}=0%
  \textbf{#2}%
  \else\ifnum\pdf@strcmp{#1}{textit}=0%
  \textit{#2}%
  \else\ifnum\pdf@strcmp{#1}{textsc}=0%
  \textsc{#2}%
  \else\ifnum\pdf@strcmp{#1}{textscth}=0%
  \textscth{#2}%
  \else\ifnum\pdf@strcmp{#1}{textth}=0%
  \thColor{#2}%
  \else\ifnum\pdf@strcmp{#1}{basic}=0%
  #2%
  \fi\fi\fi\fi\fi\fi\fi\fi\fi\fi\fi%
}
\makeatother

%%---------------------------------------------------------------%%
%                                                                 %
%                      HANDLING OF HEADERS                        %
%                        Module : Common                          %
%                                                                 %
%%---------------------------------------------------------------%%


% Apply fancy headers and footers
% And clear previously defined headers and footers
% First part, for global pages
	\pagestyle{fancy}
	\renewcommand{\sectionmark}[1]{\markright{#1}}
	\renewcommand{\chaptermark}[1]{\markboth{#1}{}}
	\fancyhead{}
	\fancyfoot{}

	% Set footer
	\fancyfoot[C]{\thepage}

	% Set header
	\fancyhead[L]{\leftmark~\raisebox{.15em}{\tiny$\blacksquare$}~\rightmark}
	\fancyhead[R]{\Parttitle}

	% Set rule width for headers and footers
	\renewcommand{\headrulewidth}{0pt}% Default \headrulewidth is 0.4pt
	\renewcommand{\footrulewidth}{0pt}% Default \footrulewidth is 0pt

	
% Second par, for plain page such as chapter, TOC...
\fancypagestyle{plain}{%
	\pagestyle{fancy}
	\fancyhead{}
	\fancyfoot{}

	% Set footer
	\fancyfoot[C]{\thepage}

	% Set rule width for headers and footers
	\renewcommand{\headrulewidth}{0pt}% Default \headrulewidth is 0.4pt
	\renewcommand{\footrulewidth}{0pt}% Default \footrulewidth is 0pt
}
\makeindex
%% Defining our itemize bullet
\newcommand\squareBullet{\textcolor{\currentColor}{\raisebox{.15em}{\tiny$\blacksquare$}}}
%\newcommand\squareBullet{\textcolor{\currentColor}{\raisebox{.15em}{\tiny$\blacksquare$}}}

\newcommand\itemperso[1]{\item[\squareBullet]~\textit{\textcolor{\currentColor}{#1}}\ifthenelse{\equal{#1}{}}{}{\quad\enskip}}
\newcommand\subitemperso[1]{\item[\squareBullet]~\textit{\textcolor{\currentColor}{#1}}\ifthenelse{\equal{#1}{}}{}{\quad\enskip}}

%\newcommand{\tabitem}{~~\llap{\mybullet}~~}
% changemargin environment
%
% Shift margins for a given part of text.
%	- Negative values : increase margins
%	- Positives values : decrease margins
%
% Parameters :
%	- offset left : measure value
%	- offset right : measure value
\def\changemargin#1#2{\list{}{\rightmargin#2\leftmargin#1}\item[]}
\let\endchangemargin=\endlist 


%%-------------------------------------------------------------------%%
%                                                                     %
%           STYLE COLLECTIONS FOR SECTION, CHAPTER...NAMES            %
%                                                                     %
%%-------------------------------------------------------------------%%


% Basic style to fully customize section name, figure and separator.
% No other fioriture than given configuration will be used.
%
% Parameters :
%	- Chapter number
%	- Chapter name
%
% Name for configuration : basic
\newcommand\sectionBasic[2]{\textbf{#1}~~\textbf{#2}}


% Style with big figure and section name pushed to the right.
%
% Parameters :
%	- Chapter number
%	- Chapter name
%
% Name for configuration : numberInSquare
\newcommand\sectionNumberInSquare[2]{%
	\hspace*{\sectionLeftSpace}\textcolor{\sectionNumberColor}{\fbox{\textbf{#1}}}~~\textcolor{\sectionNameColor}{\textbf{#2}}
}


% Chapter style with big figure and section name after.
% The whole is underlined by a thin line.
%
% Parameters :
%	- Chapter number
%	- Chapter name
%
% Name for configuration : greatNumber
\newcommand\sectionGreatNumber[2]{%
  \noindent%
  \fontsize{30}{1}\selectfont%
  \textit{\Huge{%
    \textbf{#1.~~}%
  }}
  \fontsize{15}{1}\selectfont%
  \textit{\textbf{#2}}\par%
}


% Chapter style with big figure and chapter name pushed to the right.
% The whole is underlined by a thin line.
%
% Parameters :
%	- Chapter number
%	- Chapter name
%
% Name for configuration : numberInMargin
\newcommand\sectionNumberInMargin[2]{
	\begin{changemargin}{-1cm}{0cm}\textbf{#1}~~\textbf{#2}\end{changemargin}
	\vspace*{-.25cm}
}


% Chapter style with big figure and chapter name pushed to the right.
% The whole is underlined by a thin line.
%
% Parameters :
%	- Chapter number
%	- Chapter name
%
% Name for configuration : bigFigureAndRule
\newcommand\sectionBigFigureAndRule[2]{
	\begin{flushright}
		\textbf{\fontsize{70}{60}\selectfont #1}
		\vspace*{1cm}

		\textbf{\fontsize{30}{40}\selectfont #2}

		\rule{13cm}{1pt}
	\end{flushright}
	\vspace*{1.5cm}
}


% Chapter style with big figure and chapter name pushed to the right.
% Both are separated with a vertical line.
%
% TODO: Finish this style
%
% Parameters :
%	- Chapter number
%	- Chapter name
%
% Name for configuration : bigFigureAndVerticalRule
\newcommand\sectionBigFigureAndVerticalRule[2]{
	\begin{flushright}
		\textbf{\fontsize{70}{60}\selectfont #1}~~~~\rule[-1cm]{3pt}{3cm}~~~~\textbf{\fontsize{30}{40}\selectfont #2}

		\rule{13cm}{1pt}
	\end{flushright}
}


% Similar to chapterBigFigureAndVerticalRule but with a uniform currentColor background.
% Moreover, chapter number and names are alignes on the right.
%
% TODO: Finish this style
%
% Parameters :
%	- Chapter number
%	- Chapter name
%
% Name for configuration : bigFigureAndBackground
\newcommand\sectionBigFigureColoredBackground[2]{
	\begin{changemargin}{-1.5cm}{-1.5cm}
		\noindent\textcolor{-red!75!green!50}{\rule{\paperwidth}{8cm}}\hspace*{-3cm}\textcolor{white}{\rule{1.5pt}{7cm}}\newline
	\end{changemargin}
}


% Chapter style with background made with an image. The chapter number and name
% are located centered beneath the image, inside a cartridge.
%
% Parameters :
%	- Chapter number
%	- Chapter name
%
% Name for configuration : backgroundImage
\newcommand\sectionBackgroundImage[2]{
	\vspace*{-2.8cm}%
	\begin{changemargin}{-1.5cm}{-1.5cm}
		\noindent\begin{tikzpicture}%
		  \pgfdeclarelayer{background}%
		  \pgfdeclarelayer{foreground}%
		  \pgfsetlayers{background,foreground}%
		  \begin{pgfonlayer}{foreground}%
			\node (separator) [color=blue, fill=blue, draw, rectangle, minimum width=21cm, inner sep=1pt] at (-10.5,0) {};%
			\node (section_name) [fill=white, font=\fontsize{25}{1}\selectfont, inner sep=.3cm, text depth=.35ex] at (separator.center) {\textcolor{blue}{#1~~\raisebox{2pt}{$\bullet$}}~~\textbf{#2}\hspace*{.2cm}};%
			\fill [white] ([xshift=-2pt]section_name.north east) -- (section_name.north east) -- ([xshift=.5cm]section_name.east) -- (section_name.south east) --([xshift=-2pt]section_name.south east) -- cycle ;%
			\fill [white] ([xshift=2pt]section_name.north west) -- (section_name.north west) -- ([xshift=-.5cm]section_name.west) -- (section_name.south west)  --([xshift=2pt]section_name.south west) -- cycle ;%
			\draw [color=blue, line width=2pt] (section_name.north east) -- ([xshift=.5cm]section_name.east) -- (section_name.south east) -- (section_name.south west) -- ([xshift=-.5cm]section_name.west) -- (section_name.north west) -- cycle;%
		  \end{pgfonlayer}%
		  \begin{pgfonlayer}{background}%
			\path [fill tile image*={width=\paperwidth}{images/default4.JPG}] (-21,10) rectangle (0,0);%
		  \end{pgfonlayer}%
		\end{tikzpicture}%
	\end{changemargin}
}


% Chapter style with background made with an image. The chapter number and name
% are located top left in a transparent full text width line. It also auto include
% a partial TOC in a cartridge in the right.
%
% Parameters :
%	- Chapter number
%	- Chapter name
%
% Name for configuration : backgroundImageTransparencyWithTOC
\newcommand\sectionBackgroundImageTransparency[2]{
	\vspace*{-2.8cm}%
	\begin{changemargin}{-1.5cm}{-1.5cm}
		\noindent\begin{tikzpicture}%
		  \pgfdeclarelayer{background}%
		  \pgfdeclarelayer{foreground}%
		  \pgfsetlayers{background,foreground}%
		  \begin{pgfonlayer}{foreground}%
			\node (separator) [color=white, fill=blue, opacity=0, draw, rectangle, minimum width=21cm, inner sep=1pt] at (-10.5,0) {};%
			\node (section_name) [fill=white, opacity=.55, text opacity=1, text width=\paperwidth, anchor=west, font=\fontsize{25}{1}\selectfont, inner sep=.4cm, text depth=.35ex] at (-21,8.25) {\thColor{\textbf{#1. #2}}};%
			\node (section_toc)  [rounded corners, fill=white, draw=blue, fill opacity=.55, text opacity=1, minimum height=7cm, text width=.55\paperwidth, anchor=north west, inner sep=.2cm] at (-12.25,6.25) {\normalsize\begin{minipage}{\linewidth}\startcontents\printcontents{l}{1}{\setcounter{tocdepth}{1}}\end{minipage}};%
		  \end{pgfonlayer}%
		  \begin{pgfonlayer}{background}%
			\path [fill tile image*={width=\paperwidth}{images/default4.JPG}] (-21,10) rectangle (0,0);%
		  \end{pgfonlayer}%
		\end{tikzpicture}%
	\end{changemargin}
	\vskip .5cm
}


% Chapter style with background made with an image. The chapter number and name
% are located top left in a transparent full text width line. Partial TOC is not
% auto-included with this shapre
%
% Parameters :
%	- Chapter number
%	- Chapter name
%
% Name for configuration : backgroundImageTransparencyNoTOC
\newcommand\sectionBackgroundImageTransparencyWithoutToc[2]{
	\vspace*{-2.8cm}%
	\begin{changemargin}{-1.5cm}{-1.5cm}
		\noindent\begin{tikzpicture}%
		  \pgfdeclarelayer{background}%
		  \pgfdeclarelayer{foreground}%
		  \pgfsetlayers{background,foreground}%
		  \begin{pgfonlayer}{foreground}%
			\node (separator) [color=white, fill=blue, opacity=0, draw, rectangle, minimum width=21cm, inner sep=1pt] at (-10.5,0) {};%
			\node (section_name) [fill=white, opacity=.55, text opacity=1, text width=\paperwidth, anchor=west, font=\fontsize{25}{1}\selectfont, inner sep=.4cm, text depth=.35ex] at (-21,8.25) {\thColor{\textbf{#1. #2}}};%
		  \end{pgfonlayer}%
		  \begin{pgfonlayer}{background}%
			\path [fill tile image*={width=\paperwidth}{images/default4.JPG}] (-21,10) rectangle (0,0);%
		  \end{pgfonlayer}%
		\end{tikzpicture}%
	\end{changemargin}
}
\newcommand*\Parttitle{}
\let\origpart\part
\renewcommand*{\part}[2][]{%
   \ifx\\#1\\% optional argument not present?
      \origpart{#2}%
      \renewcommand*\Parttitle{#2}%
   \else
      \origpart[#1]{#2}%
      \renewcommand*\Parttitle{#1}%
   \fi
}
%%---------------------------------------------------------------------------%%
%                                                                             %
%                                CHAPTER STYLING                              %
%                                Module : Common                              %
%                                                                             %
%%---------------------------------------------------------------------------%%


% Remove all spaces before chapter in order to align it with its expected vertical position.
% Use styles that are compatible with these values for optimal rendering.
\titlespacing*{\chapter}{0em}{0em}{0em}[0em]


%
% Title format of numbered chapters. Based on applySectionFormat method.
%
\titleformat{\chapter}
  {\fontsize{17pt}{0pt}\selectfont}
  {}
  {0em}
  {%
    \applySectionFormat{\chapterStyle}%
    {\sectionNumberFormating{\chapterNumberStyle}{\chapterNumberSize}{\chapterNumberColor}{\thechapter}}%
    {\sectionSeparatorFormating{\chapterSeparatorStyle}{\chapterSeparatorSize}{\chapterSeparatorColor}{\chapterSeparator}}%
    {\sectionNameFormating{\chapterNameStyle}{\chapterNameSize}{\chapterNameColor}{#1}}%
  }


%
% Title format of unnumbered chapters. Based on applySectionFormat method.
%
\titleformat{name=\chapter,numberless}
  {\fontsize{17pt}{0pt}\selectfont}
  {}
  {0em}
  {%
    \applySectionFormat{\chapterStyle}%
    {}%
    {}%
    {\sectionNameFormating{\chapterNameStyle}{\chapterNameSize}{\chapterNameColor}{#1}}%
  }


%
% Chapters partial toc style
%
\titlecontents{lsection}[.2cm]
  {\normalsize}{\thColor{\bfseries\thecontentslabel}~~~~}{}
  {\hfill\thecontentspage}
\contentsmargin{.2cm}
% Customize chapter display for our needs.
% We basically remove all spaces before chapter in order to align it with its expected vertical position.
% All chapter implementations expect this behaviour.
\titleformat{\section}  
  {\fontsize{17pt}{0pt}\selectfont}
  {}
  {0em}
  {%
    \applySectionFormat{\sectionStyle}%
      {\sectionNumberFormating{\sectionNumberStyle}{\sectionNumberSize}{\sectionNumberColor}{\thesection}}%
      {\sectionSeparatorFormating{\sectionSeparatorStyle}{\sectionSeparatorSize}{\sectionSeparatorColor}{\sectionSeparator}}%
      {\sectionNameFormating{\sectionNameStyle}{\sectionNameSize}{\sectionNameColor}{#1}}%
  }

\renewcommand\thesection{\arabic{section}}
% Customize chapter display for our needs.
% We basically remove all spaces before chapter in order to align it with its expected vertical position.
% All chapter implementations expect this behaviour.
\titleformat{\subsection}
  {\fontsize{15pt}{0pt}\selectfont}{}{0em}
  {\applySectionFormat{\subsectionStyle}{\thesubsection}{#1}}
\titlespacing*{\chapter}{0em}{0em}{0em}[0em]
% Customize chapter display for our needs.
% We basically remove all spaces before chapter in order to align it with its expected vertical position.
% All chapter implementations expect this behaviour.
\titleformat{\subsubsection}
  {\fontsize{13pt}{0pt}\selectfont}{}{0em}
  {\stepcounter{subsubsection}\subsubsectionBasic{\thesubsubsection}{#1}}

\renewcommand\thesubsubsection{\arabic{section}.\arabic{subsection}.\arabic{subsubsection}}
% Block paragraph with a background color and a left colored bar.
% The left bar is 2mm width and located in the margin.
%
% Parameters :
%	- Background color
%	- Left bar color
%	- Block name (optional)
%	- Block subname (optional, used only if block name is given)
\newenvironment{paragraphWithLeftBar}[4]
{
	\renewcommand{\currentColor}{#2}
	\begin{tcolorbox}[breakable,enhanced,colback=#1,boxrule=0pt,
					  frame hidden,borderline west={2mm}{0mm}{#2}]
	\ifstrempty{#3}{}{\textbf{\textcolor{#2}{#3} \ifstrempty{#4}{}{(#4)}}~~}
}
{
	\end{tcolorbox}
	\renewcommand{\currentColor}{themeColor}
}


% Application of paragraphWithLeftBar for theorems. Numerotation is automatic.
%
% Parameters :
%	- Theorem name
\newenvironment{theoreme}[1]
{\begin{paragraphWithLeftBar}{yellow!10!white}{vertforet}{Théorème}{#1}}
{\end{paragraphWithLeftBar}}

% Application of paragraphWithLeftBar for propositions. Numerotation is automatic.
%
% Parameters :
%	- Proposition name
\newenvironment{proposition}[1]
{\begin{paragraphWithLeftBar}{yellow!10!white}{bluesword}{Proposition}{#1}}
{\end{paragraphWithLeftBar}}

% Application of paragraphWithLeftBar for definitions. Numerotation is automatic.
%
% Parameters :
%	- Definition name
\newenvironment{definition}[1]
{\begin{paragraphWithLeftBar}{yellow!10!white}{bordeau}{Définition}{#1}}
{\end{paragraphWithLeftBar}}

% Application of paragraphWithLeftBar for lemmas. Numerotation is automatic.
%
% Parameters :
%	- Lemma name
\newenvironment{lemme}[1]
{\begin{paragraphWithLeftBar}{yellow!10!white}{violet}{Lemme}{#1}}
{\end{paragraphWithLeftBar}}

\newenvironment{introductoryParagraph}
{\begin{changemargin}{1cm}{1cm}\itshape}
{\end{changemargin}}

\newenvironment{exemple}[1]
{\vskip .2cm\par\noindent\textbf{Exemple -- }\textit{#1}\begin{changemargin}{1cm}{0cm}\setlist{leftmargin=1.5cm}}
{\end{changemargin}}

\newenvironment{attention}
{\vskip .2cm\par\noindent\textcolor{bordeau}{\ding{54}~\textbf{Attention~~~~}}}
{}

\newenvironment{remarque}
{\vskip .2cm\par\noindent\textcolor{bluesword}{\faCommentingO~\textbf{Remarque~~~~}}}
{}

\newenvironment{convention}
{\vskip .2cm\par\noindent\textcolor{vertforet}{\ding{46}~\textbf{Convention~~~~}}}
{}

\newenvironment{casParticulier}[1]
{\begin{paragraphWithLeftBar}{yellow!10!white}{violet}{Cas particulier}{#1}}
{\end{paragraphWithLeftBar}}

\newenvironment{methode}[1]
{\vskip .2cm\par\noindent\textcolor{vertforet}{\ding{46}~\textbf{Méthode \ifstrempty{#1}{}{(#1)}}~~~~}}
{}

\newenvironment{lecture}
{\vskip .2cm\par\noindent\textcolor{vertforet}{\faBook~\textbf{Lecture~~~~}}}
{}

\newenvironment{information}
{\vskip .2cm\par\noindent\textcolor{vertforet}{\faBook~\textbf{Information~~~~}}}
{}

\newenvironment{notation}
{\vskip .2cm\par\noindent\textcolor{vertforet}{\faBook~\textbf{Notation~~~~}}}
{}

\newenvironment{demonstration}
{\vskip .2cm\par\noindent\textbf{Démonstration~~~~}}
{\proved{}\vskip .2cm}

\newenvironment{paragraphe}[1]
{\vskip .2cm\par\noindent\textbf{#1~~~~}}
{}

\setlength{\parindent}{0em}
% Colum type for systems
\newcolumntype{o}{@{}>{{}}c<{{}}@{}}

% Rectif indices
\newcommand{\rectif}{\hspace{0.05cm}}
\newcommand{\domain}[1]{\mathcal{D}\ifstrempty{#1}{}{_{#1}}}
\newcommand{\coordplan}[2]{\left(\begin{aligned}#1\\#2\end{aligned}\right)}
\newcommand{\coordesp}[3]{\left(\begin{aligned}#1\\#2\\#3\end{aligned}\right)}
\newcommand{\expara}[1]{\newline\noindent\textbf{#1}\hspace{.4cm}}
\newcommand{\condn}{\expara{Condition nécessaire}}
\newcommand{\conds}{\expara{Condition suffisante}}
\newcommand{\analyse}{\expara{Analyse}}
\newcommand{\ranalyse}{\expara{Retour à l'analyse}}
\newcommand{\synthese}{\expara{Synthèse}}
\newcommand{\rsynthese}{\expara{Retour à la synthèse}}
\newcommand{\conclusion}{\expara{Conclusion}}
\newcommand{\pconclusion}{\expara{Conclusion partielle}}
\newcommand{\xxi}[2]{\displaystyle{\left(#1;\vv{#2}\right)}}
\newcommand{\oi}{\xxi{O}{\imath}}
\newcommand{\xij}[3]{\displaystyle{\left(#1;\vv{#2},\vv{#3}\right)}}
\newcommand{\oij}{\xij{O}{\imath}{\jmath}}
\newcommand{\iij}{\xij{I}{\imath}{\jmath}}
\newcommand{\xijk}[4]{\displaystyle{\left(#1;\vv{#2},\vv{#3},\vv{#4}\right)}}
\newcommand{\oijk}{\xijk{O}{\imath}{\jmath}{k}}
\newcommand{\glabel}[1]{\mathscr{#1}}
\newcommand{\gplabel}[1]{(\glabel{#1})}
\newcommand{\coordi}[1]{#1\vv{\imath}}
\newcommand{\coordj}[1]{#1\vv{\jmath}}
\newcommand{\coordk}[1]{#1\vv{k}}
\newcommand{\coordp}[2]{\displaystyle{\left(#1~;~#2\right)}}
\newcommand{\coorde}[3]{\displaystyle{\left(#1~;~#2~;~#3\right)}}
\newcommand{\coordee}[3]{\displaystyle{#1\vv{\imath}+#2\vv{\jmath}+#3\vv{k}}}
\newcommand{\suite}[3]{\displaystyle{\left(#1_#2\right)_{#2\in\mathbb{#3}}}}
\newcommand{\suiteu}{\displaystyle{\left(u_n\right)_{n\in\mathbb{N}}}}
\newcommand{\suitev}{\displaystyle{\left(v_n\right)_{n\in\mathbb{N}}}}
\newcommand{\txvar}[2]{\displaystyle{\lim_{h\rightarrow0}\frac{#1(#2+h)-#1(#2)}{h}}}
\newcommand{\llim}[3]{\displaystyle{\lim_{#1\rightarrow#2}#3}}
\newcommand{\lllim}[4]{\displaystyle{\lim_{\substack{#1\rightarrow#2\\#3}}#4}}
\newcommand{\tendto}[3]{\displaystyle{#1\xrightarrow[#2]{}#3}}
\newcommand{\ttendto}[4]{\displaystyle{#1\xrightarrow[#2\to#3]{}#4}}
\newcommand{\ip}{+\infty}
\newcommand{\im}{-\infty}
\newcommand{\ipm}{\pm\infty}
\newcommand{\ch}{\textnormal{ch}}
\newcommand{\sh}{\textnormal{sh}}
\newcommand{\tth}{\textnormal{th}}
\newcommand{\argch}{\textnormal{argch}}
\newcommand{\argsh}{\textnormal{argsh}}
\newcommand{\argth}{\textnormal{argth}}
\newcommand{\vect}[1]{\vv{#1}}
\newcommand{\trans}[1]{t_{\vv{#1}}}
\newcommand{\homot}[2]{h_{#1,#2}}
\newcommand{\rectvect}[2]{\displaystyle{\left[\begin{aligned}&#1\\&#2\end{aligned}\right]}}
\newcommand{\aavect}[4]{\displaystyle{\left(#1\vv{#2},#3\vv{#4}\right)}}
\newcommand{\vu}{\vect{u}}
\newcommand{\vvv}{\vect{v}}
\newcommand{\vw}{\vect{w}}
\newcommand{\avect}[2]{\aavect{}{#1}{}{#2}}
\newcommand{\mathspace}[2]{\hspace{#1}\textnormal{#2}\hspace{#1}}
\newcommand{\mathet}[1]{\mathspace{#1}{et}}
\newcommand{\mathssi}[1]{\mathspace{#1}{ssi}}
\newcommand{\mathalors}[1]{\mathspace{#1}{alors}}
\newcommand{\mathavec}[1]{\mathspace{#1}{avec}}
\newcommand{\mathie}[1]{\mathspace{#1}{ie.}}
\newcommand{\mathou}[1]{\mathspace{#1}{ou}}
\newcommand{\mathdonc}[1]{\mathspace{#1}{donc}}
\newcommand{\mathequiv}[1]{\mathspace{#1}{$\Longleftrightarrow$}}
\newcommand{\mathimply}[1]{\mathspace{#1}{$\Rightarrow$}}
\newcommand{\mathequivsub}[2]{\mathspace{#1}{$\underset{#2}{\Longleftrightarrow}$}}
\newcommand{\infequiv}[3]{\displaystyle{#1\underset{#2}{\sim}#3}}
\newcommand{\cont}[1]{\mathcal{C}^{#1}}
\newcommand{\continue}[3]{\cont{#1}(#2,#3)}
\newcommand{\cinf}{\cont{\infty}}
\newcommand{\eqnlabel}[1]{\mathcal{#1}}
\newcommand{\eqn}[1]{(\eqnlabel{#1})}
\newcommand{\eqnp}[2]{(\eqnlabel{#1}_{#2})}
\newcommand{\eqnt}[2]{(\eqnlabel{#1}_{\textnormal{#2}})}
\newcommand{\eqnsol}[2]{\eqnlabel{#1}_{#2}}
\newcommand{\eqnsolt}[2]{\eqnlabel{#1}_{\textnormal{#2}}}
\newcommand{\eqna}[1]{\displaystyle{\begin{aligned}#1\end{aligned}}}
\newcommand{\syst}[1]{\displaystyle{\left\{\begin{aligned}#1\end{aligned}\right.}}
\newcommand{\systa}[2]{\displaystyle{\left\{\begin{array}{#1}#2\end{array}\right.}}
\newcommand{\exercice}[2]{\begin{paragraphe}{Exercice #1}#2\end{paragraphe}}
\newcommand{\exerciceds}[2]{\begin{paragraphe}{Exercice #1}\vspace{.15cm}\newline#2\end{paragraphe}}
\newcommand{\norm}[1]{\left\lVert\rule{0cm}{.25cm}#1\right\rVert}
\newcommand{\vnorm}[1]{\norm{\vect{u}}}
\newcommand{\prodscal}[2]{\vect{#1}\cdot\vect{#2}}
\newcommand{\fctdisp}[5]{\displaystyle{#1:\left\{\begin{aligned}#2&\rightarrow#3\\#4&\mapsto#5\end{aligned}\right.}}
\newcommand{\fctdispd}[7]{\displaystyle{#1:\left\{\begin{aligned}#2&\rightarrow#3\\#4&\mapsto#5\\#6&\mapsto#7\end{aligned}\right.}}
\newcommand{\fctdispsmall}[3]{\displaystyle{#1:#2\rightarrow#3}}
\newcommand{\fctdispsmallb}[3]{\displaystyle{#2\overset{#1}{\rightarrow}#3}}
\newcommand{\fctdispsmallm}[3]{\displaystyle{#1:#2\mapsto#3}}
\newcommand{\fctdispminimal}[2]{\displaystyle{#1\mapsto#2}}
\newcommand{\curvenn}[2]{\mathscr{#1}_{#2}}
\newcommand{\tddet}[4]{\displaystyle{\left|\begin{array}{rr}#1&#2 \\ #3&#4\end{array}\right|}}
\newcommand{\overbar}[1]{\mkern 1.5mu\overline{\mkern-1.5mu#1\mkern-1.5mu}\mkern 1.5mu}
\newcommand{\rep}[1]{\displaystyle{\mathbf{Re}\left(#1\right)}}
\newcommand{\imp}[1]{\displaystyle{\mathbf{Im}\left(#1\right)}}
\newcommand{\zarg}[1]{\displaystyle{\mathbf{Arg}\left(#1\right)}}
\newcommand{\textenum}[3]{\displaystyle{#1}, \displaystyle{#2}\textnormal{ et }\displaystyle{#3}}
\newcommand{\textenumq}[4]{\displaystyle{#1}, \displaystyle{#2}, \displaystyle{#3}\textnormal{ et }\displaystyle{#4}}
\newcommand{\dropsign}[1]{\smash{\llap{\raisebox{-.5\normalbaselineskip}{$#1$\hspace{2\arraycolsep}}}}}
\newcommand{\equationlabel}[3]{\begin{equation}#1\label{#2}\tag{$#3$}\end{equation}}
\newcommand{\questlabel}[1]{\textcolor{\themeColor}{(#1)}}
\newcommand{\rrp}{\mathbb{R}_+}
\newcommand{\rr}{\mathbb{R}}
\newcommand{\rrd}{\mathbb{R}^2}
\newcommand{\rrm}{\mathbb{R}_-}
\newcommand{\rrpe}{\mathbb{R}_+^*}
\newcommand{\rrme}{\mathbb{R}_-^*}
\newcommand{\rre}{\mathbb{R}^*}
\newcommand{\pp}{\mathbb{P}}
\newcommand{\cc}{\mathbb{C}}
\newcommand{\cce}{\mathbb{C}^*}
\newcommand{\ccd}{\mathbb{C}^2}
\newcommand{\nn}{\mathbb{N}}
\newcommand{\nne}{\mathbb{N}^*}
\newcommand{\nnd}{\mathbb{N}^2}
\newcommand{\zz}{\mathbb{Z}}
\newcommand{\zzd}{\mathbb{Z}^2}
\newcommand{\zze}{\mathbb{Z}^*}
\newcommand{\qq}{\mathbb{Q}}
\newcommand{\uu}{\mathbb{U}}
\newcommand{\uui}[1]{\mathbb{U}_{#1}}
\newcommand{\kk}{\mathbb{K}}
\newcommand{\pge}[1]{\textnormal{PGE}(#1)}
\newcommand{\pgcd}[2]{#1\land#2}
\newcommand{\ppcm}[2]{#1\lor#2}
\newcommand{\comp}[2]{#1\circ#2}
\newcommand{\compt}[3]{#1\circ#2\circ#3}
\newcommand{\comptt}[2]{#2\circ#1\circ#2^{-1}}
\newcommand{\fctid}[1]{\textnormal{id}_{#1}}
\newcommand{\proved}{\hspace*{\fill}$\blacksquare$}
\newcommand{\perm}[1]{\mathfrak{S}(#1)}
\newcommand{\conj}[2]{#1\sim#2}
\newcommand{\parent}[1]{\left(#1\right)}
\newcommand{\intcroch}[3]{\displaystyle{\left[#1\right]_{#2}^{#3}}}
\newcommand{\zconj}[1]{\overbar{#1}}
\newcommand{\congr}[1]{\hspace{.15cm}\left[#1\right]}
\newcommand{\congru}[3]{#1\equiv#2\congr{#3}}
\newcommand{\ncongru}[3]{#1\not\equiv#2\congr{#3}}
\newcommand{\card}[1]{\textnormal{card}\left(#1\right)}
\newcommand{\dist}[2]{\textnormal{dist}\coordp{#1}{#2}}
\newcommand{\entint}[2]{\left\llbracket#1;#2\right\rrbracket}
\newcommand{\intint}[4]{\displaystyle{\left#1#2;#3\right#4}}
\newcommand{\closeint}[2]{\intint{[}{#1}{#2}{]}}
\newcommand{\tq}{\hspace{.1cm}\large/\hspace{.1cm}}
\newcommand{\cotan}{\textnormal{cotan}}
\newcommand{\cotanh}{\textnormal{cotanh}}
\newcommand{\ddfrac}[2]{\frac{\displaystyle{#1}}{\displaystyle{#2}}}
\newcommand*{\diffdchar}{\textnormal{d}}
\newcommand*{\dd}{\mathop{\diffdchar\!}}
\newcommand{\der}[1]{\displaystyle{\frac{\dd}{\dd#1}}}
\newcommand{\derp}[2]{\displaystyle{\frac{\dd{}^{#2}}{\dd#1^{#2}}}}
\newcommand{\dder}[2]{\displaystyle{\frac{\dd#1}{\dd#2}}}
\newcommand{\dderp}[3]{\displaystyle{\frac{\dd{}^{#3}#1}{\dd#2^{#3}}}}
\newcommand{\integrale}[2]{\displaystyle{\int#1\,\diffdchar#2}}
\newcommand{\ecrbase}[2]{\overbar{#1}^{(#2)}}

\DeclareMathAlphabet\mathbfcal{OMS}{cmsy}{b}{n}

\newcommand{\fctname}[1]{\mathbfcal{#1}}
\newcommand{\angps}[2]{\left\langle#1,#2\right\rangle}
\newcommand*{\transp}[1]{^t\!#1}%
\newcommand{\entpartname}{\mathbfcal{E}}
\newcommand{\entpart}[1]{\displaystyle{\entpartname\left(#1\right)}}
\newcommand{\complexitealg}[1]{\displaystyle{\mathbfcal{O}\left(#1\right)}}
\newcommand{\complexitealgsub}[2]{\displaystyle{\mathbfcal{O}_{\textnormal{#2}}\left(#1\right)}}


% In-text version of previous commands
\newcommand{\Resp}{respectivement }
\newcommand{\Domain}[1]{$\domain{#1}$}
\newcommand{\Coordplan}[2]{$\displaystyle{\coordplan{1}{2}}$}
\newcommand{\Coordesp}[3]{$\displaystyle{\coordesp{1}{2}{3}}$}
\newcommand{\Xxi}[2]{$\xxi{1}{2}$}
\newcommand{\Oi}{$\xxi{O}{\imath}$}
\newcommand{\Xij}[3]{$\displaystyle{\left(#1;\vv{#2},\vv{#3}\right)}$}
\newcommand{\Oij}{$\oij{}$}
\newcommand{\Iij}{$\iij{}$}
\newcommand{\Xijk}[4]{$\xijk{#1}{#2}{#3}{#4}$}
\newcommand{\Oijk}{$\oijk{}$}
\newcommand{\Glabel}[1]{$\glabel{#1}$}
\newcommand{\Gplabel}[1]{$\gplabel{#1}$}
\newcommand{\Coordi}[1]{$\displaystyle{\coordi{#1}}$}
\newcommand{\Coordj}[1]{$\displaystyle{\coordj{#1}}$}
\newcommand{\Coordk}[1]{$\displaystyle{\coordk{#1}}$}
\newcommand{\Coordp}[2]{$\coordp{#1}{#2}$}
\newcommand{\Coorde}[3]{$\coorde{#1}{#2}{#3}$}
\newcommand{\Coordee}[3]{$\coordee{#1}{#2}{#3}$}
\newcommand{\Suite}[3]{$\suite{#1}{#2}{#3}$}
\newcommand{\Suiteu}{$\suiteu{}$}
\newcommand{\Suitev}{$\suitev{}$}
\newcommand{\Txvar}[2]{$\txvar{1}{2}$}
\newcommand{\Llim}[3]{$\llim{#1}{#2}{#3}$}
\newcommand{\Lllim}[4]{$\lllim{#1}{#2}{#3}{#4}$}
\newcommand{\Tendto}[3]{$\tendto{#1}{#2}{#3}$}
\newcommand{\Ttendto}[4]{$\ttendto{#1}{#2}{#3}{#4}$}
\newcommand{\Ip}{$\ip{}$}
\newcommand{\IM}{$\im{}$}
\newcommand{\Ipm}{$\ipm{}$}
\newcommand{\Ch}{$\ch{}$}
\newcommand{\Sh}{$\sh{}$}
\newcommand{\Tth}{$\tth{}$}
\newcommand{\Argch}{$\argch{}$}
\newcommand{\Argsh}{$\argsh{}$}
\newcommand{\Argth}{$\argth{}$}
\newcommand{\Vect}[1]{$\vv{#1}$}
\newcommand{\Trans}[1]{$\trans{#1}$}
\newcommand{\Homot}[2]{$\homot{#1}{#2}$}
\newcommand{\Rectvect}[2]{$\rectvect{#1}{#2}$}
\newcommand{\Aavect}[4]{$\aavect{#1}{#2}{#3}{#4}$}
\newcommand{\Vu}{$\vu{}$}
\newcommand{\Vvv}{$\vvv{}$}
\newcommand{\Vw}{$\vw{}$}
\newcommand{\Avect}[2]{$\avect{#1}{#2}$}
\newcommand{\Mathspace}[2]{$\mathspace{#1}{#2}$}
\newcommand{\Mathet}[1]{$\mathet{#1}$}
\newcommand{\Mathssi}[1]{$\mathssi{#1}$}
\newcommand{\Mathalors}[1]{$\mathalors{#1}$}
\newcommand{\Mathavec}[1]{$\mathavec{#1}$}
\newcommand{\Mathie}[1]{$\mathie{#1}$}
\newcommand{\Mathou}[1]{$\mathou{#1}$}
\newcommand{\Mathdonc}[1]{$\mathdonc{#1}$}
\newcommand{\Mathequiv}[1]{$\mathequiv{#1}$}
\newcommand{\Mathimply}[1]{$\mathimply{#1}$}
\newcommand{\Mathequivsub}[2]{$\mathequivsub{#1}{#2}$}
\newcommand{\Infequiv}[3]{$\infequiv{#1}{#2}{#3}$}
\newcommand{\Cont}[1]{$\cont{#1}$}
\newcommand{\Continue}[3]{$\continue{#1}{#2}{#3}$}
\newcommand{\Cinf}{$\cinf$}
\newcommand{\Eqnlabel}[1]{$\eqnlabel{#1}$}
\newcommand{\Eqn}[1]{$\eqn{#1}$}
\newcommand{\Eqnp}[2]{$\eqnp{#1}{#2}$}
\newcommand{\Eqnt}[2]{$\eqnt{#1}{#2}$}
\newcommand{\Eqnsol}[2]{$\eqnsol{#1}{#2}$}
\newcommand{\Eqnsolt}[2]{$\eqnsolt{#1}{#2}$}
\newcommand{\Eqna}[1]{$\eqna{#1}$}
\newcommand{\Syst}[1]{$\syst{#1}$}
\newcommand{\Systa}[2]{$\systa{#1}{#2}$}
\newcommand{\Norm}[1]{$\norm{#1}$}
\newcommand{\Vnorm}[1]{$\vnorm{#1}$}
\newcommand{\Prodscal}[2]{$\prodscal{#1}{#2}$}
\newcommand{\Fctdisp}[5]{$\fctdisp{#1}{#2}{#3}{#4}{#5}$}
\newcommand{\Fctdispd}[7]{$\fctdispd{#1}{#2}{#3}{#4}{#5}{#6}{#7}$}
\newcommand{\Fctdispsmall}[3]{$\fctdispsmall{#1}{#2}{#3}$}
\newcommand{\Fctdispsmallb}[3]{$\fctdispsmallb{#1}{#2}{#3}$}
\newcommand{\Fctdispsmallm}[3]{$\fctdispsmallm{#1}{#2}{#3}$}
\newcommand{\Fctdispminimal}[2]{$\fctdispminimal{#1}{#2}$}
\newcommand{\Curvenn}[2]{$\curvenn{#1}{#2}$}
\newcommand{\Tddet}[4]{$\tddet{#1}{#2}{#3}{#4}$}
\newcommand{\Overbar}[1]{$\overbar{#1}$}
\newcommand{\Rep}[1]{$\rep{#1}$}
\newcommand{\Imp}[1]{$\imp{#1}$}
\newcommand{\Zarg}[1]{$\zarg{#1}$}
\newcommand{\Textenum}[3]{$\textenum{#1}{#2}{#3}$}
\newcommand{\Textenumq}[4]{$\textenumq{#1}{#2}{#3}{#4}$}
\newcommand{\Equationlabel}[3]{$\equationlabel{#1}{#2}{#3}$}
\newcommand{\Rrp}{$\rrp{}$}
\newcommand{\Rr}{$\rr{}$}
\newcommand{\Rrd}{$\rrd{}$}
\newcommand{\Rrm}{$\rrm{}$}
\newcommand{\Rrpe}{$\rrpe{}$}
\newcommand{\Rrme}{$\rrme$}
\newcommand{\Rre}{$\rre{}$}
\newcommand{\Pp}{$\pp{}$}
\newcommand{\Cc}{$\cc{}$}
\newcommand{\Cce}{$\cce{}$}
\newcommand{\Ccd}{$\ccd{}$}
\newcommand{\Nn}{$\nn{}$}
\newcommand{\Nne}{$\nne{}$}
\newcommand{\Nnd}{$\nnd{}$}
\newcommand{\Zz}{$\zz{}$}
\newcommand{\Zzd}{$\zzd{}$}
\newcommand{\Zze}{$\zze{}$}
\newcommand{\Qq}{$\qq{}$}
\newcommand{\Uu}{$\uu{}$}
\newcommand{\Uui}[1]{$\uu{#1}$}
\newcommand{\Kk}{$\kk{}$}
\newcommand{\Pge}[1]{$\pge{#1}$}
\newcommand{\Pgcd}[2]{$\pgcd{#1}{#2}$}
\newcommand{\Ppcm}[2]{$\ppcm{#1}{#2}$}
\newcommand{\Comp}[2]{$\comp{#1}{#2}$}
\newcommand{\Compt}[3]{$\compt{#1}{#2}{#3}$}
\newcommand{\Comptt}[2]{$\comptt{#1}{#2}$}
\newcommand{\Fctid}[1]{$\fctid{#1}$}
\newcommand{\Perm}[1]{$\perm{#1}$}
\newcommand{\Conj}[2]{$\conj{#1}{#2}$}
\newcommand{\Parent}[1]{$\parent{#1}$}
\newcommand{\Intcroch}[3]{$\intcroch{#1}{#2}{#3}$}
\newcommand{\Zconj}[1]{$\zconj{#1}$}
\newcommand{\Congr}[1]{$\congr{#1}$}
\newcommand{\Congru}[3]{$\congru{#1}{#2}{#3}$}
\newcommand{\Ncongru}[3]{$\ncongru{#1}{#2}{#3}$}
\newcommand{\Card}[1]{$\card{#1}$}
\newcommand{\Dist}[2]{$\dist{#1}{#2}$}
\newcommand{\Entint}[2]{$\entint{#1}{#2}$}
\newcommand{\Intint}[4]{$\intint{#1}{#2}{#3}{#4}$}
\newcommand{\Closeint}[2]{$\closint{#1}{#2}$}
\newcommand{\Cotan}{$\cotan{}$}
\newcommand{\Cotanh}{$\cotanh{}$}
\newcommand{\Ddfrac}[2]{$\ddfrac{#1}{#2}$}
\newcommand*{\Diffdchar}{$\diffchar{}$}
\newcommand*{\Dd}{$\dd{}$}
\newcommand{\Der}[1]{$\der{#1}$}
\newcommand{\Derp}[2]{$\derp{#1}{#2}$}
\newcommand{\Dder}[2]{$\dder{#1}{#2}$}
\newcommand{\Dderp}[3]{$\dderp{#1}{#2}{#3}$}
\newcommand{\Integrale}[2]{$\integrale{#1}{#2}$}
\newcommand{\Ecrbase}[2]{$\ecrbase{#1}{#2}$}
\newcommand{\Dctname}[1]{$\dctname{#1}$}
\newcommand{\Angps}[2]{$\angps{#1}{#2}$}
\newcommand*{\Transp}[1]{$\transp{#1}$}%
\newcommand{\Entpartname}{$\entpartname{}$}
\newcommand{\Entpart}[1]{$\entpart{#1}$}
\newcommand{\Complexitealg}[1]{$\complexitealg{#1}$}
\newcommand{\Complexitealgsub}[2]{$\complexitealgsub{#1}{#2}$}
%%-------------------------------------------------------------------%%
%                                                                     %
%                          STYLE FOR FIGURES                          %
%                                                                     %
%%-------------------------------------------------------------------%%


% Figure caption style
\DeclareCaptionLabelSeparator{bullet}{~\headColor{$\bullet$}~}
\captionsetup{labelfont={bf,color=\headersColor},labelsep=bullet}


% Compact way of adding figures in text. Do not use this with tikz (TeX) figures.
%
% Parameters:
%	- size parameters
%	- figure location
%	- caption
%	- label
\newcommand{\figurePerso}[4]{\begin{figure}[!ht]\centering\includegraphics[#1]{#2}\caption{#3}\label{#4}\end{figure}}


% Compact way of adding TikZ figure and span over full document width.
%
% Parameters:
%	- figure location
%	- caption
%	- label
\newcommand{\figurePersoTikz}[3]{\begin{figure}[h]\makebox[\textwidth][c]{\input{#1}}\caption{#2}\label{#3}\end{figure}}


% Compact way of adding TikZ figure and center it along page width
%
% Parameters:
%	- figure location
%	- caption
%	- label
\newcommand{\figurePersoTikzz}[3]{\begingroup\centering\input{#1}\captionof{figure}{#2}\label{#3}\endgroup}


% Compact way of adding TikZ figure without caption and centering it
%
% Parameters:
%	- figure location
\newcommand{\figurePersoTikzSansCaption}[1]{\begingroup\centering\input{#1}\endgroup}


% Include image for lstlisting and centering vertically on line.
%
% Parameters:
%	- figure location
%	- height of image
\newcommand{\vCenterLstlisting}[2]{\begingroup
\setbox0=\hbox{\includegraphics[height=#2]{#1}}%
\parbox{\wd0}{\box0}\endgroup}
\definecolor{codegreen}{rgb}{0,0.6,0}
\definecolor{codegray}{rgb}{0.5,0.5,0.5}
\definecolor{codepurple}{rgb}{0.58,0,0.82}
\definecolor{backcolour}{rgb}{0.95,0.95,0.92}

\lstdefinestyle{mystyle}{%
    backgroundcolor=\color{black},%
    commentstyle=\color{codegreen}\itshape,%
    keywordstyle=\color{magenta},%
    numberstyle=\tiny\bfseries\color{white},%
    stringstyle=\color{codepurple},%
    basicstyle=\small\linespread{1}\ttfamily\color{white},%
    breakatwhitespace=false,%
    breaklines=true,%
    numbers=left,%
    numbersep=5pt,%
    showspaces=false,%
    showstringspaces=false,%
    showtabs=false,%
    tabsize=2,%
	framesep=0pt,%
	framerule=0pt,%
	framexleftmargin=.5cm,%
	xleftmargin=.5cm,%
	aboveskip=0cm%
}

\lstset{style=mystyle}


% General environment to embed code. Will create a fullwidth
% listing env, adapted to given language (ie. with logo, name...)
% 
% Parameters:
%	- location of language logo 
%	- desired height of language logo
%	- language name
%	- code block name (given by user)
\newcommand\generalCodeCaption[4]{%
	\noindent\setlength{\fboxsep}{2pt}\newline\fcolorbox{gray!15}{gray!15}{\begin{minipage}{\dimexpr\textwidth-2\fboxsep-2\fboxrule\relax}~\vCenterLstlisting{#1}{#2}~~#3~~--~~#4\end{minipage}}\vspace{-1pt}%
}


% Environment to embed Python 3 code. Will create a
% fullwidth listing env.
%
% Params :
%	- Code block name
\lstnewenvironment{python3}[1]
{
	\generalCodeCaption{images/python.eps}{1.2em}{Python 3.x}{#1}%
	\lstset{language=python, style=mystyle}%
}
{}


% Environment to embed Python code. Will create a
% fullwidth listing env.
%
% Params :
%	- Code block name
\lstnewenvironment{python}[1]
{
	\generalCodeCaption{images/python.eps}{1.2em}{Python}{#1}%
	\lstset{language=python, style=mystyle}%
}
{}


% Environment to embed JAVA code. Will create a
% fullwidth listing env.
%
% Params :
%	- Code block name
\lstnewenvironment{java}[1]
{
	\generalCodeCaption{images/java.eps}{1.6em}{JAVA}{#1}%
	\lstset{language=python, style=mystyle}%
}
{}


% Environment to embed shell code. Will create a
% fullwidth listing env.
%
% Params :
%	- Code block name
\lstnewenvironment{shell}[1]
{
	\generalCodeCaption{images/java.eps}{1.6em}{Bash}{#1}%
	\lstset{language=bash, style=mystyle}%
}
{}


% Environment to embed shell code. Will create a
% fullwidth listing env.
%
% Params :
%	- Code block name
\lstnewenvironment{wincmd}[1]
{
	\generalCodeCaption{images/java.eps}{1.6em}{Command line}{#1}%
	\lstset{language=bash, style=mystyle}%
}
{}
%%---------------------------------------------------------------------------%%
%                                                                             %
%               APPLY SECTION FORMATS FROM VALUES IN CONFIGURATION            %
%                                Module : Common                              %
%                                                                             %
%%---------------------------------------------------------------------------%%


% Apply a section format by its name as given in the configuration
% of this common module.
% 
% Parameters :
% - Name of the style
% - Section number with style
% - Section separator with style
% - Section label with style
\makeatletter
\newcommand{\applySectionFormat}[4]{%
  \ifnum\pdf@strcmp{#1}{default}=0%
  \sectionBasic{#2}{#3}{#4}%
  \else\ifnum\pdf@strcmp{#1}{defaultNoNumber}=0%
  \sectionBasicNoNumber{#2}{#3}{#4}%
  \else\ifnum\pdf@strcmp{#1}{numberInSquare}=0%
  \sectionNumberInSquare{#2}{#3}{#4}%
  \else\ifnum\pdf@strcmp{#1}{numberInSquareNoNumber}=0%
  \sectionNumberInSquareNoNumber{#2}{#3}{#4}%
  \else\ifnum\pdf@strcmp{#1}{greatNumber}=0%
  \sectionGreatNumber{#2}{#3}{#4}%
  \else\ifnum\pdf@strcmp{#1}{greatNumberNoNumber}=0%
  \sectionGreatNumberNoNumber{#2}{#3}{#4}%
  \else\ifnum\pdf@strcmp{#1}{numberInMargin}=0%
  \sectionNumberInMargin{#2}{#3}{#4}%
  \else\ifnum\pdf@strcmp{#1}{numberInMarginNoNumber}=0%
  \sectionNumberInMarginNoNumber{#2}{#3}{#4}%
  \else\ifnum\pdf@strcmp{#1}{bigFigureAndRule}=0%
  \sectionBigFigureAndRule{#2}{#3}{#4}%
  \else\ifnum\pdf@strcmp{#1}{bigFigureAndRuleNoNumber}=0%
  \sectionBigFigureAndRuleNoNumber{#2}{#3}{#4}%
  \else\ifnum\pdf@strcmp{#1}{bigFigureAndVerticalRule}=0%
  \sectionBigFigureAndVerticalRule{#2}{#3}{#4}%
  \else\ifnum\pdf@strcmp{#1}{bigFigureAndVerticalRuleNoNumber}=0%
  \sectionBigFigureAndVerticalRuleNoNumber{#2}{#3}{#4}%
  \else\ifnum\pdf@strcmp{#1}{bigFigureAndBackground}=0%
  \sectionBigFigureColoredBackground{#2}{#3}{#4}%
  \else\ifnum\pdf@strcmp{#1}{bigFigureAndBackgroundNoNumber}=0%
  \sectionBigFigureColoredBackgroundNoNumber{#2}{#3}{#4}%
  \else\ifnum\pdf@strcmp{#1}{backgroundImage}=0%
  \sectionBackgroundImage{#2}{#3}{#4}%
  \else\ifnum\pdf@strcmp{#1}{backgroundImageNoNumber}=0%
  \sectionBackgroundImageNoNumber{#2}{#3}{#4}%
  \else\ifnum\pdf@strcmp{#1}{backgroundImageTransparencyWithTOC}=0%
  \sectionBackgroundImageTransparency{#2}{#3}{#4}%
  \else\ifnum\pdf@strcmp{#1}{backgroundImageTransparencyWithTOCNoNumber}=0%
  \sectionBackgroundImageTransparencyNoNumber{#2}{#3}{#4}%
  \else\ifnum\pdf@strcmp{#1}{backgroundImageTransparencyNoTOC}=0%
  \sectionBackgroundImageTransparencyWithoutToc{#2}{#3}{#4}%
  \else\ifnum\pdf@strcmp{#1}{backgroundImageTransparencyNoTOCNoNumber}=0%
  \sectionBackgroundImageTransparencyWithoutTocNoNumber{#2}{#3}{#4}%
  \fi\fi\fi\fi\fi\fi\fi\fi\fi\fi\fi\fi\fi\fi\fi\fi\fi\fi\fi\fi
}
\makeatother


% Formatting of section number style from configurations
%
% Parameters :
% - Text style
% - Text size
% - Text color
% - Section number
\newcommand{\sectionNumberFormating}[4]{%
  \textcolor{#3}{\applyTextStyle{#1}{#4}}%
}


% Formatting of section separator style from configurations
%
% Parameters :
% - Text style
% - Text size
% - Text color
% - Separator
\newcommand{\sectionSeparatorFormating}[4]{%
  \textcolor{#3}{\applyTextStyle{#1}{#4}}
}


% Formatting of section separator style from configurations
%
% Parameters :
% - Text style
% - Text size
% - Text color
% - Section name
\newcommand{\sectionNameFormating}[4]{%
  \textcolor{#3}{\applyTextStyle{#1}{#4}}
}
%%---------------------------------------------------------------------------%%
%                                                                             %
%                   GLOBAL CONFIGURATIONS FOR COMMON MODULES                  %
%                                Module : Common                              %
%                                                                             %
%%---------------------------------------------------------------------------%%


% Styles for sectionning
%
% To view all style names see : common_module/common__section_styles.tex
\newcommand{\partStyle}{}
\newcommand{\chapterStyle}{}
\newcommand{\sectionStyle}{}
\newcommand{\subsectionStyle}{}
\newcommand{\subsubsectionStyle}{}


% Styles for colors
%
%  To view all defined colors see : common_module/common__colors.tex
\newcommand{\themeColor}{bordeau}                    % Main color for the document
\newcommand{\currentColor}{\themeColor}              % Save a color to use it in an area
\newcommand{\headersColor}{\themeColor}              % Color for headers


%% Include additional configurations for common module
%%---------------------------------------------------------------------------%%
%                                                                             %
%                ADDITIONAL CONFIGURATIONS FOR SECTION STYLES                 %
%                                Module : Common                              %
%                                                                             %
%%---------------------------------------------------------------------------%%

%% Part styles
\newcommand{\partNumberColor}{\themeColor}
\newcommand{\partNameColor}{\themeColor}

%% Chapter styles
\newcommand{\chapterNumberStyle}{\themeColor}
\newcommand{\chapterNumberSize}{\themeColor}
\newcommand{\chapterNumberColor}{\themeColor}
\newcommand{\chapterSeparator}{\themeColor}
\newcommand{\chapterSeparatorStyle}{\themeColor}
\newcommand{\chapterSeparatorSize}{\themeColor}
\newcommand{\chapterSeparatorColor}{\themeColor}
\newcommand{\chapterNameStyle}{\themeColor}
\newcommand{\chapterNameSize}{\themeColor}
\newcommand{\chapterNameColor}{\themeColor}
\newcommand{\chapterLeftSpace}{0pt}

%% Section styles
\newcommand{\sectionNumberStyle}{\themeColor}
\newcommand{\sectionNumberSize}{\themeColor}
\newcommand{\sectionNumberColor}{\themeColor}
\newcommand{\sectionSeparator}{\themeColor}
\newcommand{\sectionSeparatorStyle}{\themeColor}
\newcommand{\sectionSeparatorSize}{\themeColor}
\newcommand{\sectionSeparatorColor}{\themeColor}
\newcommand{\sectionNameStyle}{\themeColor}
\newcommand{\sectionNameSize}{\themeColor}
\newcommand{\sectionNameColor}{\themeColor}
\newcommand{\sectionLeftSpace}{0pt}

%% Subsection styles
\newcommand{\subsectionNumberStyle}{\themeColor}
\newcommand{\subsectionNumberSize}{\themeColor}
\newcommand{\subsectionNumberColor}{\themeColor}
\newcommand{\subsectionSeparator}{\themeColor}
\newcommand{\subsectionSeparatorStyle}{\themeColor}
\newcommand{\subsectionSeparatorSize}{\themeColor}
\newcommand{\subsectionSeparatorColor}{\themeColor}
\newcommand{\subsectionNameStyle}{\themeColor}
\newcommand{\subsectionNameSize}{\themeColor}
\newcommand{\subsectionNameColor}{\themeColor}
\newcommand{\subsectionLeftSpace}{0pt}

%% Subsubection styles
\newcommand{\subsubsectionNumberStyle}{\themeColor}
\newcommand{\subsubsectionNumberSize}{\themeColor}
\newcommand{\subsubsectionNumberColor}{\themeColor}
\newcommand{\subsubsectionSeparator}{\themeColor}
\newcommand{\subsubsectionSeparatorStyle}{\themeColor}
\newcommand{\subsubsectionSeparatorSize}{\themeColor}
\newcommand{\subsubsectionSeparatorColor}{\themeColor}
\newcommand{\subsubsectionNameStyle}{\themeColor}
\newcommand{\subsubsectionNameSize}{\themeColor}
\newcommand{\subsubsectionNameColor}{\themeColor}
\newcommand{\subsubsectionLeftSpace}{0pt}
\usepackage{fancyvrb}
\usepackage{pbox}
\usepackage[utf8]{inputenc}

\newcommand*{\fvtextcolor}[2]{\textcolor{#1}{#2}}
\CustomVerbatimCommand{\inCodeStub}{Verb}{commandchars=§?!}

%% Beginning of the file
\begin{document}

{\Large
\begin{center}
DOCUMENTATION DU TEMPLATE DE DOCUMENT
\vspace*{.4cm}
\end{center}
}

\section{Les couleurs}
\subsection{Cadencier de couleur}
Ce template fournit par défaut un certain nombre de couleurs. Cette page présente un cadencier de toutes les couleurs prédéfinies. Ces dernières sont composées de la manière suivantes :
\begin{itemize}
	\itemperso{}Des couleurs de bases.
	\itemperso{}Des déclinaisons de ces couleurs virant de plus en plus vers le blanc.
	\itemperso{}Des couleurs solitaires sans déclinaison.
\end{itemize}
\begin{multicols}{4}
	\noindent\testColors{green1}%
	\testColors{green3}\testColors{green4}%
	\testColors{red1}\testColors{red2}\testColors{red3}\testColors{red4}\testColors{red5}\testColors{red6}\testColors{red7}\testColors{red8}\testColors{red9}\testColors{red10}\testColors{red11}%
	\testColors{blue1}\testColors{blue2}\testColors{blue3}\testColors{blue4}\testColors{blue5}\testColors{blue6}\testColors{blue7}\testColors{blue8}\testColors{blue9}\testColors{blue10}\testColors{blue11}%
	\testColors{yellow1}\testColors{yellow2}\testColors{yellow3}\testColors{yellow4}\testColors{yellow5}\testColors{yellow6}\testColors{yellow7}\testColors{yellow8}\testColors{yellow9}\testColors{yellow10}\testColors{yellow11}%
	\testColors{brown1}%
	\testColors{violet1}\testColors{violet2}\testColors{violet3}\testColors{violet4}\testColors{violet5}\testColors{violet6}\testColors{violet7}\testColors{violet8}\testColors{violet9}\testColors{violet10}\testColors{violet11}%
	\testColors{orange1}\testColors{orange2}\testColors{orange3}\testColors{orange4}\testColors{orange5}\testColors{orange6}\testColors{orange7}\testColors{orange8}\testColors{orange9}\testColors{orange10}\testColors{orange11}%
	\testColors{pink1}\testColors{pink2}\testColors{pink3}\testColors{pink4}\testColors{pink5}\testColors{pink6}\testColors{pink7}\testColors{pink8}\testColors{pink9}\testColors{pink10}\testColors{pink11}%
	\testColors{gray1}\testColors{gray2}\testColors{gray3}\testColors{gray4}\testColors{gray5}\testColors{gray6}\testColors{gray7}\testColors{gray8}\testColors{gray9}\testColors{gray10}\testColors{gray11}%
	\end{multicols}
\subsection{Gestion des couleurs dans le document}
Le document utilise un certain nombre de couleurs prédéfinies, qui peuvent être paramétrées dans le fichier XX :
\begin{itemize}
	\itemperso{\texttt{thColor}}La couleur de base du document. Il s'agit de la couleur par défaut utilisée pour colorer les éléments nécessitant de la couleur.
	\itemperso{\texttt{sectionColor}}La couleur pour le libellé des sections, sous-sections et sous-sous sections.
	\itemperso{\texttt{sectionNumberColor}}La couleur pour le numéro des sections, sous-sections et sous-sous sections.
	\itemperso{\texttt{sectionSeparatorColor}}La couleur pour le séparateur entre le numéro de sections, sous-sections et sous-sous section et son libellé.
\end{itemize}

\section{Les différents types d'écritures}
Nous listons ici toutes les commandes par défaut permettant de réaliser la mise en forme du texte. Ces commandes viennent en support des commandes traditionnelles pour l'emphase (\inCodeStub¡§fvtextcolor?bordeau!?textbf!¡, \inCodeStub¡§fvtextcolor?bordeau!?textit!¡).

\section{Les différents types de paragraphes}
\subsection{Remarque générale}
Par paragraphe, nous entendons ici le style de certaines boîtes de texte pour mettre en avant du contenu. Ce type est défini de manière globale pour tout le document. Ainsi, en parant d'un choix parmi les différents types possibles (section XX), ce choix sera appliqué à tous les paragraphes pré-définis (section XX) et offrira une commande pour créer de nouveaux paragraphes personnalisés basés sur ce dernier.

\subsection{Les différents types de paragraphes}


\subsection{Les paragraphes pré-définis}
\begin{theoreme}{Fondamental de l'arithmétique}
Ceci est un théorème
\end{theoreme}

\begin{proposition}{Caractérisation des fonctions croissantes}
Ceci est une proposition
\end{proposition}

\begin{definition}{Fonction croissante}
Ceci est une définition
\end{definition}

\section{Les différents types de section}

\iffalse
\resizebox{.5\textwidth}{!}{%
\vspace*{2.8cm}
\chapterBackgroundImage{1}{Les nombres complexes}
}

\resizebox{.5\textwidth}{!}{%
\vspace*{2.8cm}
\chapterBackgroundImageTransparencyWithoutToc{1}{Les nombres complexes}
}

\resizebox{.5\textwidth}{!}{%
\vspace*{2.8cm}
\chapterBackgroundImageTransparency{1}{Les nombres complexes}
}

\resizebox{.5\textwidth}{!}{%
\chapterBigFigureAndVerticalRule{1}{Les nombres complexes}
}

\resizebox{.5\textwidth}{!}{%
\chapterBigFigureAndRule{1}{Les nombres complexes}
}

\resizebox{.5\textwidth}{!}{
\vspace*{2.8cm}
\chapterBigFigureColoredBackground{1}{Les nombres complexes}
}
\fi

\section{Commandes pour les environnements mathématiques}
\subsection{Remarque générale}
Toutes les commandes présentées ici nécessitent d'être réalisées dans un environnement mathématique \inCodeStub¡§fvtextcolor?bordeau!?$...$!¡ ou \inCodeStub¡§fvtextcolor?bordeau!?\[...\]!¡, sauf mention du contraire.

Pour toutes les commandes suivies du symbole \textcolor{bordeau}{\texttt{*}}, une version hors environnement mathématique existe. Cette version prend les mêmes paramètres en entrée, mais la première lettre de son nom est à mettre en majuscule.

\subsection{Coordonnées}

%%
%% Coordplan
%%
\vspace*{.75cm}

\inCodeStub¡§fvtextcolor?bordeau!?\coordplan{!<x>§fvtextcolor?bordeau!?}{!<y>§fvtextcolor?bordeau!?}!¡ -- Coordonnées dans le plan en colonne.

\setlength{\leftskip}{.75cm}%
\setlength{\textwidth}{17.25cm}%

\colorbox{blue!15}{\pbox{.25\textwidth}{$\coordplan{\sqrt{x}}{\dfrac{1}{x^2}}$}}
\hfill
\begin{minipage}{.65\textwidth}
	\begin{lstlisting}[linewidth=\textwidth, language={[LaTeX]TeX}]
	$\coordplan{\sqrt{x}}{\dfrac{1}{x^2}}$
	\end{lstlisting}
\end{minipage}

\setlength{\leftskip}{0pt}
\setlength{\textwidth}{18cm}%


%%
%% Coordp
%%
\vspace*{.75cm}

\inCodeStub¡§fvtextcolor?bordeau!?\coordp{!<x>§fvtextcolor?bordeau!?}{!<y>§fvtextcolor?bordeau!?}!¡ -- Coordonnées dans le plan en ligne.

\setlength{\leftskip}{.75cm}%
\setlength{\textwidth}{17.25cm}%

\colorbox{blue!15}{\pbox{.25\textwidth}{$\coordp{\sqrt{x}}{\dfrac{1}{x^2}}$}}
\hfill
\begin{minipage}{.65\textwidth}
	\begin{lstlisting}[linewidth=\textwidth, language={[LaTeX]TeX}]
	$\coordp{\sqrt{x}}{\dfrac{1}{x^2}}$
	\end{lstlisting}
\end{minipage}

\setlength{\leftskip}{0pt}
\setlength{\textwidth}{18cm}%


%%
%% Coordesp
%%
\vspace*{.75cm}

\inCodeStub¡§fvtextcolor?bordeau!?\coordesp{!<x>§fvtextcolor?bordeau!?}{!<y>§fvtextcolor?bordeau!?}{!<z>§fvtextcolor?bordeau!?}!¡ -- Coordonnées dans l'espace en colonne.

\setlength{\leftskip}{.75cm}%
\setlength{\textwidth}{17.25cm}%

\colorbox{blue!15}{\pbox{.25\textwidth}{$\coordesp{\sqrt{x}}{\dfrac{1}{x^2}}{2x}$}}
\hfill
\begin{minipage}{.65\textwidth}
	\begin{lstlisting}[linewidth=\textwidth, language={[LaTeX]TeX}]
	$\coordesp{\sqrt{x}}{\dfrac{1}{x^2}}{2x}$
	\end{lstlisting}
\end{minipage}

\setlength{\leftskip}{0pt}
\setlength{\textwidth}{18cm}%


%%
%% Coorde
%%
\vspace*{.75cm}

\inCodeStub¡§fvtextcolor?bordeau!?\coorde{!<x>§fvtextcolor?bordeau!?}{!<y>§fvtextcolor?bordeau!?}{!<z>§fvtextcolor?bordeau!?}!¡ -- Coordonnées dans l'espace en ligne.

\setlength{\leftskip}{.75cm}%
\setlength{\textwidth}{17.25cm}%

\colorbox{blue!15}{\pbox{.25\textwidth}{$\coorde{\sqrt{x}}{\dfrac{1}{x^2}}{2x}$}}
\hfill
\begin{minipage}{.65\textwidth}
	\begin{lstlisting}[linewidth=\textwidth, language={[LaTeX]TeX}]
	$\coorde{\sqrt{x}}{\dfrac{1}{x^2}}{2x}$
	\end{lstlisting}
\end{minipage}

\setlength{\leftskip}{0pt}
\setlength{\textwidth}{18cm}%


%%
%% Coordee
%%
\vspace*{.75cm}

\inCodeStub¡§fvtextcolor?bordeau!?\coordee{!<x>§fvtextcolor?bordeau!?}{!<y>§fvtextcolor?bordeau!?}{!<z>§fvtextcolor?bordeau!?}!¡ -- Coordonnées dans l'espace selon $\vv{\imath}$, $\vv{\jmath}$ et $\vv{k}$.

\setlength{\leftskip}{.75cm}%
\setlength{\textwidth}{17.25cm}%

\colorbox{blue!15}{\pbox{.25\textwidth}{$\coordee{\sqrt{x}}{\dfrac{1}{x^2}}{2x}$}}
\hfill
\begin{minipage}{.65\textwidth}
	\begin{lstlisting}[linewidth=\textwidth, language={[LaTeX]TeX}]
	$\coordee{\sqrt{x}}{\dfrac{1}{x^2}}{2x}$
	\end{lstlisting}
\end{minipage}

\setlength{\leftskip}{0pt}
\setlength{\textwidth}{18cm}%


\subsection{Intervalles}
\subsection{Noms de vecteurs}
\subsection{Opérations vectorielles}
\subsection{Noms de fonctions}
\subsection{Opérations sur les fonctions}
\subsection{Dérivation}
\subsection{Intégration}
\subsection{Informations textuelles}
Ces diverses commandes ne sont pas à utiliser dans un environnement mathématique, mais directement dans le texte.

%%
%% Expara
%%
\vspace*{.75cm}

\inCodeStub¡§fvtextcolor?bordeau!?\expara{!<name>§fvtextcolor?bordeau!?}!¡ -- Nom de paragraphe pour un exercice

\setlength{\leftskip}{.75cm}%
\setlength{\textwidth}{17.25cm}%

\colorbox{blue!15}{\pbox{.25\textwidth}{$\expara{Étude d'une fonction}$}}
\hfill
\begin{minipage}{.65\textwidth}
	\begin{lstlisting}[linewidth=\textwidth, language={[LaTeX]TeX}]
	$\expara{Étude d'une fonction}$
	\end{lstlisting}
\end{minipage}

Différentes applications sont faites de cette commande pour des noms standards.

\colorbox{blue!15}{\pbox{.25\textwidth}{$\condn{}\conds{}\analyse{}\ranalyse{}\synthese{}\rsynthese{}\conclusion{}\pconclusion{}$}}
\hfill
\begin{minipage}{.65\textwidth}
	\begin{lstlisting}[linewidth=\textwidth, language={[LaTeX]TeX}]
	$\condn{}$
	$\conds{}$
	$\analyse{}$
	$\ranalyse{}$
	$\synthese{}$
	$\rsynthese{}$
	$\conclusion{}$
	$\pconclusion{}$
	\end{lstlisting}
\end{minipage}

\setlength{\leftskip}{0pt}
\setlength{\textwidth}{18cm}%







%%
%% XXi
%%
\vspace*{.75cm}

\inCodeStub¡§fvtextcolor?bordeau!?\xxi{!<origine>§fvtextcolor?bordeau!?}{!<nom vecteur>§fvtextcolor?bordeau!?}!¡ -- Afficher le nom d'un repère sur une droite

\setlength{\leftskip}{.75cm}%
\setlength{\textwidth}{17.25cm}%

\colorbox{blue!15}{\pbox{.25\textwidth}{$\xxi{X}{AB}$}}
\hfill
\begin{minipage}{.65\textwidth}
	\begin{lstlisting}[linewidth=\textwidth, language={[LaTeX]TeX}]
	$\xxi{X}{AB}$
	\end{lstlisting}
\end{minipage}

Différentes applications sont faites de cette commande pour des noms standards.

\colorbox{blue!15}{\pbox{.25\textwidth}{$\oi{}$}}
\hfill
\begin{minipage}{.65\textwidth}
	\begin{lstlisting}[linewidth=\textwidth, language={[LaTeX]TeX}]
	$\oi{}$
	\end{lstlisting}
\end{minipage}

\setlength{\leftskip}{0pt}
\setlength{\textwidth}{18cm}%


%%
%% Xij
%%
\vspace*{.75cm}

\inCodeStub¡§fvtextcolor?bordeau!?\xij{!<origine>§fvtextcolor?bordeau!?}{!<vecteur x>§fvtextcolor?bordeau!?}{!<vecteur y>§fvtextcolor?bordeau!?}!¡ -- Afficher le nom d'un repère dans un plan

\setlength{\leftskip}{.75cm}%
\setlength{\textwidth}{17.25cm}%

\colorbox{blue!15}{\pbox{.25\textwidth}{$\xij{X}{AB}{AC}$}}
\hfill
\begin{minipage}{.65\textwidth}
	\begin{lstlisting}[linewidth=\textwidth, language={[LaTeX]TeX}]
	$\xij{X}{AB}{AC}$
	\end{lstlisting}
\end{minipage}

Différentes applications sont faites de cette commande pour des noms standards.

\colorbox{blue!15}{\pbox{.25\textwidth}{$\oij{}$\\$\iij{}$}}
\hfill
\begin{minipage}{.65\textwidth}
	\begin{lstlisting}[linewidth=\textwidth, language={[LaTeX]TeX}]
	$\oij{}$
	$\iij{}$
	\end{lstlisting}
\end{minipage}

\setlength{\leftskip}{0pt}
\setlength{\textwidth}{18cm}%



%%
%% Xijk
%%
\vspace*{.75cm}

\inCodeStub¡§fvtextcolor?bordeau!?\xijk{!<origine>§fvtextcolor?bordeau!?}{!<vecteur x>§fvtextcolor?bordeau!?}{!<vecteur y>§fvtextcolor?bordeau!?}{!<vecteur z>§fvtextcolor?bordeau!?}!¡ -- Afficher le nom d'un repère dans l'espace

\setlength{\leftskip}{.75cm}%
\setlength{\textwidth}{17.25cm}%

\colorbox{blue!15}{\pbox{.25\textwidth}{$\xijk{X}{AB}{AC}{AD}$}}
\hfill
\begin{minipage}{.65\textwidth}
	\begin{lstlisting}[linewidth=\textwidth, language={[LaTeX]TeX}]
	$\xijk{X}{AB}{AC}{AD}$
	\end{lstlisting}
\end{minipage}

Différentes applications sont faites de cette commande pour des noms standards.

\colorbox{blue!15}{\pbox{.25\textwidth}{$\oijk{}$}}
\hfill
\begin{minipage}{.65\textwidth}
	\begin{lstlisting}[linewidth=\textwidth, language={[LaTeX]TeX}]
	$\oijk{}$
	\end{lstlisting}
\end{minipage}

\setlength{\leftskip}{0pt}
\setlength{\textwidth}{18cm}%


%%
%% Glabel
%%
\vspace*{.75cm}

\inCodeStub¡§fvtextcolor?bordeau!?\glabel{!<lettre>§fvtextcolor?bordeau!?}!¡ -- Police calligraphiée pour les graphes et les noms de courbes

\setlength{\leftskip}{.75cm}%
\setlength{\textwidth}{17.25cm}%

\colorbox{blue!15}{\pbox{.25\textwidth}{$\glabel{A}$}}
\hfill
\begin{minipage}{.65\textwidth}
	\begin{lstlisting}[linewidth=\textwidth, language={[LaTeX]TeX}]
	$\glabel{A}$
	\end{lstlisting}
\end{minipage}

\setlength{\leftskip}{0pt}
\setlength{\textwidth}{18cm}%


%%
%% Coordi
%%
\vspace*{.75cm}

\inCodeStub¡§fvtextcolor?bordeau!?\coordi{!<x>§fvtextcolor?bordeau!?}!¡ -- Coordonnées sur l'axe des abscisses selon $\vv{\imath}$.

\setlength{\leftskip}{.75cm}%
\setlength{\textwidth}{17.25cm}%

\colorbox{blue!15}{\pbox{.25\textwidth}{$\coordi{10}$}}
\hfill
\begin{minipage}{.65\textwidth}
	\begin{lstlisting}[linewidth=\textwidth, language={[LaTeX]TeX}]
	$\coordi{10}$
	\end{lstlisting}
\end{minipage}

\setlength{\leftskip}{0pt}
\setlength{\textwidth}{18cm}%


%%
%% Coordj
%%
\vspace*{.75cm}

\inCodeStub¡§fvtextcolor?bordeau!?\coordj{!<x>§fvtextcolor?bordeau!?}!¡ -- Coordonnées sur l'axe des abscisses selon $\vv{\jmath}$.

\setlength{\leftskip}{.75cm}%
\setlength{\textwidth}{17.25cm}%

\colorbox{blue!15}{\pbox{.25\textwidth}{$\coordj{10}$}}
\hfill
\begin{minipage}{.65\textwidth}
	\begin{lstlisting}[linewidth=\textwidth, language={[LaTeX]TeX}]
	$\coordj{10}$
	\end{lstlisting}
\end{minipage}

\setlength{\leftskip}{0pt}
\setlength{\textwidth}{18cm}%


%%
%% Coordk
%%
\vspace*{.75cm}

\inCodeStub¡§fvtextcolor?bordeau!?\coordk{!<x>§fvtextcolor?bordeau!?}!¡ -- Coordonnées sur l'axe des abscisses selon $\vv{k}$.

\setlength{\leftskip}{.75cm}%
\setlength{\textwidth}{17.25cm}%

\colorbox{blue!15}{\pbox{.25\textwidth}{$\coordk{10}$}}
\hfill
\begin{minipage}{.65\textwidth}
	\begin{lstlisting}[linewidth=\textwidth, language={[LaTeX]TeX}]
	$\coordk{10}$
	\end{lstlisting}
\end{minipage}

\setlength{\leftskip}{0pt}
\setlength{\textwidth}{18cm}%


%%
%% Suite
%%
\vspace*{.75cm}

\inCodeStub¡§fvtextcolor?bordeau!?\suite{!<nom>§fvtextcolor?bordeau!?}{!<variable>§fvtextcolor?bordeau!?}{!<ensemble>§fvtextcolor?bordeau!?}!¡ -- Affiche le nom d'une suite.

\setlength{\leftskip}{.75cm}%
\setlength{\textwidth}{17.25cm}%

\colorbox{blue!15}{\pbox{.25\textwidth}{$\suite{x}{n}{\nn^*}$}}
\hfill
\begin{minipage}{.65\textwidth}
	\begin{lstlisting}[linewidth=\textwidth, language={[LaTeX]TeX}]
	$\suite{x}{n}{\nn^*}$
	\end{lstlisting}
\end{minipage}

\setlength{\leftskip}{0pt}
\setlength{\textwidth}{18cm}%


%%
%% Suiteu
%%
\vspace*{.75cm}

\inCodeStub¡§fvtextcolor?bordeau!?\suiteu{}!¡ -- Raccourci pour une suite nommée $u$ définie sur $\nn$.

\setlength{\leftskip}{.75cm}%
\setlength{\textwidth}{17.25cm}%

\colorbox{blue!15}{\pbox{.25\textwidth}{$\suiteu{}$}}
\hfill
\begin{minipage}{.65\textwidth}
	\begin{lstlisting}[linewidth=\textwidth, language={[LaTeX]TeX}]
	$\suiteu{}$
	\end{lstlisting}
\end{minipage}

\setlength{\leftskip}{0pt}
\setlength{\textwidth}{18cm}%


%%
%% Suitev
%%
\vspace*{.75cm}

\inCodeStub¡§fvtextcolor?bordeau!?\suitev{}!¡ -- Raccourci pour une suite nommée $v$ définie sur $\nn$.

\setlength{\leftskip}{.75cm}%
\setlength{\textwidth}{17.25cm}%

\colorbox{blue!15}{\pbox{.25\textwidth}{$\suitev{}$}}
\hfill
\begin{minipage}{.65\textwidth}
	\begin{lstlisting}[linewidth=\textwidth, language={[LaTeX]TeX}]
	$\suitev{}$
	\end{lstlisting}
\end{minipage}

\setlength{\leftskip}{0pt}
\setlength{\textwidth}{18cm}%


%%
%% txvar
%%
\vspace*{.75cm}

\inCodeStub¡§fvtextcolor?bordeau!?\txvar{!<nom fonction>§fvtextcolor?bordeau!?}{!<nom variable>§fvtextcolor?bordeau!?}!¡ -- Affiche le taux de variation de la fonction selon une variable à prendre différente de $h$..

\setlength{\leftskip}{.75cm}%
\setlength{\textwidth}{17.25cm}%

\colorbox{blue!15}{\pbox{.25\textwidth}{$\txvar{f}{1}$}}
\hfill
\begin{minipage}{.65\textwidth}
	\begin{lstlisting}[linewidth=\textwidth, language={[LaTeX]TeX}]
	$\txvar{f}{1}$
	\end{lstlisting}
\end{minipage}

\setlength{\leftskip}{0pt}
\setlength{\textwidth}{18cm}%


%%
%% Llim
%%
\vspace*{.75cm}

\inCodeStub¡§fvtextcolor?bordeau!?\llim{!<variable>§fvtextcolor?bordeau!?}{!<limite>§fvtextcolor?bordeau!?}{!<fonction>§fvtextcolor?bordeau!?}!¡ -- Limite d'une fonction en une valeur et variable données.

\setlength{\leftskip}{.75cm}%
\setlength{\textwidth}{17.25cm}%

\colorbox{blue!15}{\pbox{.25\textwidth}{$\llim{x}{+\infty}{f(x)}$}}
\hfill
\begin{minipage}{.65\textwidth}
	\begin{lstlisting}[linewidth=\textwidth, language={[LaTeX]TeX}]
	$\llim{x}{+\infty}{f(x)}$
	\end{lstlisting}
\end{minipage}

\setlength{\leftskip}{0pt}
\setlength{\textwidth}{18cm}%


%%
%% Lllim
%%
\vspace*{.75cm}

\inCodeStub¡§fvtextcolor?bordeau!?\lllim{!<variable>§fvtextcolor?bordeau!?}{!<limite>§fvtextcolor?bordeau!?}{!<complément>§fvtextcolor?bordeau!?}{!<fonction>§fvtextcolor?bordeau!?}!¡ -- Limite d'une fonction en une valeur et variable données avec précisions sur la limite. À utiliser notament pour les limites à gauche et à droite.

\setlength{\leftskip}{.75cm}%
\setlength{\textwidth}{17.25cm}%

\colorbox{blue!15}{\pbox{.25\textwidth}{$\lllim{x}{0}{x>0}{f(x)}$}}
\hfill
\begin{minipage}{.65\textwidth}
	\begin{lstlisting}[linewidth=\textwidth, language={[LaTeX]TeX}]
	$\lllim{x}{0}{x>0}{f(x)}$
	\end{lstlisting}
\end{minipage}

\setlength{\leftskip}{0pt}
\setlength{\textwidth}{18cm}%


%%
%% Tendto
%%
\vspace*{.75cm}

\inCodeStub¡§fvtextcolor?bordeau!?\tendto{!<fonction>§fvtextcolor?bordeau!?}{!<variable>§fvtextcolor?bordeau!?}{!<valeur limite>§fvtextcolor?bordeau!?}!¡ -- Version en ligne de la limite d'une fonction en une valeur, sans préciser la variable.

\setlength{\leftskip}{.75cm}%
\setlength{\textwidth}{17.25cm}%

\colorbox{blue!15}{\pbox{.25\textwidth}{$\tendto{f(x)}{+\infty}{\sqrt{\pi}}$}}
\hfill
\begin{minipage}{.65\textwidth}
	\begin{lstlisting}[linewidth=\textwidth, language={[LaTeX]TeX}]
	$\tendto{f(x)}{+\infty}{\sqrt{\pi}}$
	\end{lstlisting}
\end{minipage}

\setlength{\leftskip}{0pt}
\setlength{\textwidth}{18cm}%


%%
%% Ttendto
%%
\vspace*{.75cm}

\inCodeStub¡§fvtextcolor?bordeau!?\ttendto{!<focntion>§fvtextcolor?bordeau!?}{!<variable>§fvtextcolor?bordeau!?}{!<limite>§fvtextcolor?bordeau!?}{!<valeur limite>§fvtextcolor?bordeau!?}!¡ -- Version en ligne de la limite d'une fonction en une valeur, en précisant la variable.

\setlength{\leftskip}{.75cm}%
\setlength{\textwidth}{17.25cm}%

\colorbox{blue!15}{\pbox{.25\textwidth}{$\ttendto{f(x,y)}{x}{+\infty}{\sqrt{\pi}}$}}
\hfill
\begin{minipage}{.65\textwidth}
	\begin{lstlisting}[linewidth=\textwidth, language={[LaTeX]TeX}]
	$\ttendto{f(x,y)}{x}{+\infty}{\sqrt{\pi}}$
	\end{lstlisting}
\end{minipage}

\setlength{\leftskip}{0pt}
\setlength{\textwidth}{18cm}%


%%
%% Ip, Im, Ipm
%%
\vspace*{.75cm}

\inCodeStub¡§fvtextcolor?bordeau!?\ip{}!, §fvtextcolor?bordeau!?\im{}!, §fvtextcolor?bordeau!?\ipm{}!¡ -- Expressions liées à l'infini.

\setlength{\leftskip}{.75cm}%
\setlength{\textwidth}{17.25cm}%

\colorbox{blue!15}{\pbox{.25\textwidth}{$\ip{}$, $\im{}$, $\ipm{}$}}
\hfill
\begin{minipage}{.65\textwidth}
	\begin{lstlisting}[linewidth=\textwidth, language={[LaTeX]TeX}]
	$\ip{}$, $\im{}$, $\ipm{}$
	\end{lstlisting}
\end{minipage}

\setlength{\leftskip}{0pt}
\setlength{\textwidth}{18cm}%


%%
%% Cotan
%%
\vspace*{.75cm}

\inCodeStub¡§fvtextcolor?bordeau!?\cotan{}!¡ -- Complément aux fonction de trigonométrie circulaire.

\setlength{\leftskip}{.75cm}%
\setlength{\textwidth}{17.25cm}%

\colorbox{blue!15}{\pbox{.25\textwidth}{$\cotan(x)$}}
\hfill
\begin{minipage}{.65\textwidth}
	\begin{lstlisting}[linewidth=\textwidth, language={[LaTeX]TeX}]
	$\cotan(x)$
	\end{lstlisting}
\end{minipage}

\setlength{\leftskip}{0pt}
\setlength{\textwidth}{18cm}%



%%
%% Ch, Sh, Tth
%%
\vspace*{.75cm}

\inCodeStub¡§fvtextcolor?bordeau!?\ch{}!, §fvtextcolor?bordeau!?\sh{}!, §fvtextcolor?bordeau!?\tth{}!, §fvtextcolor?bordeau!?\cotanh{}!¡ -- Fonctions de trigonométrie hyperbolique

\setlength{\leftskip}{.75cm}%
\setlength{\textwidth}{17.25cm}%

\colorbox{blue!15}{\pbox{.35\textwidth}{$\ch(x)$, $\sh(x)$, $\tth(x)$, $\cotanh(x)$}}
\hfill
\begin{minipage}{.65\textwidth}
	\begin{lstlisting}[linewidth=\textwidth, language={[LaTeX]TeX}]
	$\ch(x)$, $\sh(x)$, $\tth(x)$, $\cotanh(x)$
	\end{lstlisting}
\end{minipage}

\setlength{\leftskip}{0pt}
\setlength{\textwidth}{18cm}%


%%
%% Argch, Argsh, Argth
%%
\vspace*{.75cm}

\inCodeStub¡§fvtextcolor?bordeau!?\argch{}!, §fvtextcolor?bordeau!?\argsh{}!, §fvtextcolor?bordeau!?\argth{}!¡ -- Fonctions réciproques de trigonométrie hyperbolique

\setlength{\leftskip}{.75cm}%
\setlength{\textwidth}{17.25cm}%

\colorbox{blue!15}{\pbox{.35\textwidth}{$\argch(x)$, $\argsh(x)$, $\argth(x)$}}
\hfill
\begin{minipage}{.65\textwidth}
	\begin{lstlisting}[linewidth=\textwidth, language={[LaTeX]TeX}]
	$\argch(x)$, $\argsh(x)$, $\argth(x)$
	\end{lstlisting}
\end{minipage}

\setlength{\leftskip}{0pt}
\setlength{\textwidth}{18cm}%


%%
%% Vect
%%
\vspace*{.75cm}

\inCodeStub¡§fvtextcolor?bordeau!?\vect{!<nom vecteur>§fvtextcolor?bordeau!?}!¡ -- Nom d'un vecteur

\setlength{\leftskip}{.75cm}%
\setlength{\textwidth}{17.25cm}%

\colorbox{blue!15}{\pbox{.25\textwidth}{$\vect{w}$}}
\hfill
\begin{minipage}{.65\textwidth}
	\begin{lstlisting}[linewidth=\textwidth, language={[LaTeX]TeX}]
	$\vect{w}$
	\end{lstlisting}
\end{minipage}

\setlength{\leftskip}{0pt}
\setlength{\textwidth}{18cm}%


%%
%% Trans
%%
\vspace*{.75cm}

\inCodeStub¡§fvtextcolor?bordeau!?\trans{!<nom vecteur>§fvtextcolor?bordeau!?}!¡ -- Expression d'une translation

\setlength{\leftskip}{.75cm}%
\setlength{\textwidth}{17.25cm}%

\colorbox{blue!15}{\pbox{.25\textwidth}{$\trans{w}$}}
\hfill
\begin{minipage}{.65\textwidth}
	\begin{lstlisting}[linewidth=\textwidth, language={[LaTeX]TeX}]
	$\trans{w}$
	\end{lstlisting}
\end{minipage}

\setlength{\leftskip}{0pt}
\setlength{\textwidth}{18cm}%


%%
%% Homot
%%
\vspace*{.75cm}

\inCodeStub¡§fvtextcolor?bordeau!?\homot{!<centre>§fvtextcolor?bordeau!?}{!<rapport>§fvtextcolor?bordeau!?}!¡ -- Expression d'une homothétie en fonction de son centre et son rapport.

\setlength{\leftskip}{.75cm}%
\setlength{\textwidth}{17.25cm}%

\colorbox{blue!15}{\pbox{.25\textwidth}{$\homot{O}{\frac{1}{2}}$}}
\hfill
\begin{minipage}{.65\textwidth}
	\begin{lstlisting}[linewidth=\textwidth, language={[LaTeX]TeX}]
	$\homot{O}{\frac{1}{2}}$
	\end{lstlisting}
\end{minipage}

\setlength{\leftskip}{0pt}
\setlength{\textwidth}{18cm}%


%%
%% Vu, Vvv, Vw
%%
\vspace*{.75cm}

\inCodeStub¡§fvtextcolor?bordeau!?\vu{}!, §fvtextcolor?bordeau!?\vvv{}!, §fvtextcolor?bordeau!?\vw{}!¡ -- Raccourcis pour des noms classiques de vecteurs

\setlength{\leftskip}{.75cm}%
\setlength{\textwidth}{17.25cm}%

\colorbox{blue!15}{\pbox{.25\textwidth}{Les vecteurs $\vu{}$, $\vvv{}$ et $\vw{}$.}}
\hfill
\begin{minipage}{.65\textwidth}
	\begin{lstlisting}[linewidth=\textwidth, language={[LaTeX]TeX}]
	Les vecteurs $\vu{}$, $\vvv{}$ et $\vw{}$.
	\end{lstlisting}
\end{minipage}

\setlength{\leftskip}{0pt}
\setlength{\textwidth}{18cm}%


%%
%% Mathspace, Mathet, Mathssi, Mathalors, Mathavec, Mathie, Mathou, Mathdonc, Mathequiv, Mathimply
%%
\vspace*{.75cm}

\inCodeStub¡§fvtextcolor?bordeau!?\mathspace{!<taille espacement>§fvtextcolor?bordeau!?}{!<connecteur>§fvtextcolor?bordeau!?}!¡ -- Lier deux éléments d'un raisonnement par un connecteur (ou, donc, alors, $\Leftrightarrow$...) en laissant un espace de chaque côté du connecteur. Le \texttt{<connecteur>} affiché l'est en tant que texte, il faut donc se replacer en contexte mathématique (\texttt{\$...\$}) pour y afficher une formule.

\setlength{\leftskip}{.75cm}%
\setlength{\textwidth}{17.25cm}%

\colorbox{blue!15}{\pbox{.45\textwidth}{$a^2<b^2\mathspace{1cm}{$(\Rightarrow{}, \Leftarrow{})$?}a<b$.}}
\hfill
\begin{minipage}{.60\textwidth}
	\begin{lstlisting}[linewidth=\textwidth, language={[LaTeX]TeX}]
	$a^2<b^2\mathspace{1cm}{$(\Rightarrow{}, \Leftarrow{})$?}a<b$.
	\end{lstlisting}
\end{minipage}

De nombreux usages sont faits de cette commande pour définir tous les connecteurs classiques. Pour chaque redéfinition, l'appel se fait de la forme suivante : \inCodeStub¡§fvtextcolor?bordeau!?\mathconnecteur{!<espacement>§fvtextcolor?bordeau!?}!¡.

\colorbox{blue!15}{\pbox{.45\textwidth}{$a=0\mathet{1cm}b=1$\\$a=0\mathalors{1cm}b=1$\\$a=0\mathavec{1cm}b=1$\\$a=0\mathou{1cm}b=1$\\$a=0\mathdonc{1cm}b=1$\\$a=0\mathimply{1cm}b=1$\\$a\in\mathcal{E}\mathie{1cm}a\in2\zz$\\$a\in\mathcal{E}\mathssi{1cm}a\in2\zz$\\$a\in\mathcal{E}\mathequiv{1cm}a\in2\zz$}}
\hfill
\begin{minipage}{.60\textwidth}
	\begin{lstlisting}[linewidth=\textwidth, language={[LaTeX]TeX}]
	$a=0\mathet{1cm}b=1$
	$a=0\mathalors{1cm}b=1$
	$a=0\mathavec{1cm}b=1$
	$a=0\mathou{1cm}b=1$
	$a=0\mathdonc{1cm}b=1$
	$a=0\mathimply{1cm}b=1$
	$a\in\mathcal{E}\mathie{1cm}a\in2\zz$
	$a\in\mathcal{E}\mathssi{1cm}a\in2\zz$
	$a\in\mathcal{E}\mathequiv{1cm}a\in2\zz$
	\end{lstlisting}
\end{minipage}

\setlength{\leftskip}{0pt}
\setlength{\textwidth}{18cm}%


%%
%% Mathequivsub
%%
\vspace*{.75cm}

\inCodeStub¡§fvtextcolor?bordeau!?\mathequivsub{!<espacement>§fvtextcolor?bordeau!?}{!<précisions>§fvtextcolor?bordeau!?}!¡ -- Lier deux éléments de raisonnement pas un symbole $\Longleftrightarrow$ associés à une précision sous le symbôle. Utilise la commande \inCodeStub¡§fvtextcolor?bordeau!?\mathspace!¡ vue précédement.

\setlength{\leftskip}{.75cm}%
\setlength{\textwidth}{17.25cm}%

\colorbox{blue!15}{\pbox{.25\textwidth}{$a\in\mathcal{E}\mathequivsub{.75cm}{(\star)}a\in\zz$.}}
\hfill
\begin{minipage}{.65\textwidth}
	\begin{lstlisting}[linewidth=\textwidth, language={[LaTeX]TeX}]
	$a\in\mathcal{E}\mathequivsub{.75cm}{(\star)}a\in\zz$.
	\end{lstlisting}
\end{minipage}

\setlength{\leftskip}{0pt}
\setlength{\textwidth}{18cm}%


%%
%% Infequiv
%%
\vspace*{.75cm}

\inCodeStub¡§fvtextcolor?bordeau!?\infequiv{!<expression>§fvtextcolor?bordeau!?}{!<précisions>§fvtextcolor?bordeau!?}{!<équivalent>§fvtextcolor?bordeau!?}!¡ -- Équivalence d'une expression en une valeur.

\setlength{\leftskip}{.75cm}%
\setlength{\textwidth}{17.25cm}%

\colorbox{blue!15}{\pbox{.25\textwidth}{$\infequiv{f(x)}{\ip{}}{o\left(\dfrac{1}{x}\right)}$}}
\hfill
\begin{minipage}{.65\textwidth}
	\begin{lstlisting}[linewidth=\textwidth, language={[LaTeX]TeX}]
	$\infequiv{f(x)}{\ip{}}{o\left(\dfrac{1}{x}\right)}$
	\end{lstlisting}
\end{minipage}

\setlength{\leftskip}{0pt}
\setlength{\textwidth}{18cm}%


%%
%% Cont
%%
\vspace*{.75cm}

\inCodeStub¡§fvtextcolor?bordeau!?\cont{!<classe continuité>§fvtextcolor?bordeau!?}!¡ -- Classe de continuité d'une fonction.

\setlength{\leftskip}{.75cm}%
\setlength{\textwidth}{17.25cm}%

\colorbox{blue!15}{\pbox{.25\textwidth}{$\cont{1}$}}
\hfill
\begin{minipage}{.65\textwidth}
	\begin{lstlisting}[linewidth=\textwidth, language={[LaTeX]TeX}]
	$\cont{1}$
	\end{lstlisting}
\end{minipage}

\setlength{\leftskip}{0pt}
\setlength{\textwidth}{18cm}%


%%
%% Continue
%%
\vspace*{.75cm}

\inCodeStub¡§fvtextcolor?bordeau!?\continue{!<classe continuité>§fvtextcolor?bordeau!?}{!<ensemble de départ>§fvtextcolor?bordeau!?}{!<ensemble d'arrivée>§fvtextcolor?bordeau!?}!¡ -- Classe de continuité d'une fonction en précisant les ensembles d'arrivée et de départ.

\setlength{\leftskip}{.75cm}%
\setlength{\textwidth}{17.25cm}%

\colorbox{blue!15}{\pbox{.25\textwidth}{$\continue{1}{\rr}{\rr^*}$}}
\hfill
\begin{minipage}{.65\textwidth}
	\begin{lstlisting}[linewidth=\textwidth, language={[LaTeX]TeX}]
	$\continue{1}{\rr}{\rr^*}$
	\end{lstlisting}
\end{minipage}

\setlength{\leftskip}{0pt}
\setlength{\textwidth}{18cm}%


%%
%% Cinf
%%
\vspace*{.75cm}

\inCodeStub¡§fvtextcolor?bordeau!?\cinf{}!¡ -- Classe de continuité infinie pour une fonction.

\setlength{\leftskip}{.75cm}%
\setlength{\textwidth}{17.25cm}%

\colorbox{blue!15}{\pbox{.25\textwidth}{$\cinf{}$}}
\hfill
\begin{minipage}{.65\textwidth}
	\begin{lstlisting}[linewidth=\textwidth, language={[LaTeX]TeX}]
	$\cinf{}$
	\end{lstlisting}
\end{minipage}

\setlength{\leftskip}{0pt}
\setlength{\textwidth}{18cm}%


%%
%% Eqnlabel
%%
\vspace*{.75cm}

\inCodeStub¡§fvtextcolor?bordeau!?\eqnlabel{!<label>§fvtextcolor?bordeau!?}!¡ -- Libellé d'une équation.

\setlength{\leftskip}{.75cm}%
\setlength{\textwidth}{17.25cm}%

\colorbox{blue!15}{\pbox{.25\textwidth}{$\eqnlabel{E}$}}
\hfill
\begin{minipage}{.65\textwidth}
	\begin{lstlisting}[linewidth=\textwidth, language={[LaTeX]TeX}]
	$\eqnlabel{E}$
	\end{lstlisting}
\end{minipage}

Cette forme de base est ensuite réutilisée pour les différentes formes de labels d'équations, indicées ou non.

\colorbox{blue!15}{\pbox{.25\textwidth}{$\eqn{E}$\\$\eqnp{E}{1}$\\$\eqnt{E}{continuité}$\\$\eqnsol{S}{1}$\\$\eqnsolt{S}{continuité}$}}
\hfill
\begin{minipage}{.65\textwidth}
	\begin{lstlisting}[linewidth=\textwidth, language={[LaTeX]TeX}]
	$\eqn{E}$
	$\eqnp{E}{1}$
	$\eqnt{E}{continuité}$
	$\eqnsol{S}{1}$
	$\eqnsolt{S}{continuité}$
	\end{lstlisting}
\end{minipage}

\setlength{\leftskip}{0pt}
\setlength{\textwidth}{18cm}%


%%
%% \Rrp \Rr \Rrd \Rrm \Rrpe \Rrme \Rre \Pp \Cc \Cce \Ccd \Nn \Nne \Nnd \Zz \Zzd \Zze \Qq \Uu \Kk
%%
\vspace*{.75cm}

\inCodeStub¡§fvtextcolor?bordeau!?\cc{}!, §fvtextcolor?bordeau!?\cce{}!, §fvtextcolor?bordeau!?\ccd{}!, §fvtextcolor?bordeau!?\nn{}!, §fvtextcolor?bordeau!?\nne{}!, §fvtextcolor?bordeau!?\nnd{}!, §fvtextcolor?bordeau!?\qq{}!, §fvtextcolor?bordeau!?\uu{}!, §fvtextcolor?bordeau!?\kk{}!, §fvtextcolor?bordeau!?\pp{}!, §fvtextcolor?bordeau!?\rrp{}!, §fvtextcolor?bordeau!?\rr{}!¡\newline
\inCodeStub¡§fvtextcolor?bordeau!?\rrd{}!, §fvtextcolor?bordeau!?\rrm{}!, §fvtextcolor?bordeau!?\rrpe{}!, §fvtextcolor?bordeau!?\rrme{}!, §fvtextcolor?bordeau!?\rre{}!, §fvtextcolor?bordeau!?\zz{}!, §fvtextcolor?bordeau!?\zzd{}!, §fvtextcolor?bordeau!?\zze{}!¡ -- Noms d'ensembles communs.

\setlength{\leftskip}{.75cm}%
\setlength{\textwidth}{17.25cm}%

\colorbox{blue!15}{\pbox{.25\textwidth}{$\rr{}~\rre{}~\rrd{}~\rrm{}~\rrme{}~\rrp{}~\rrpe{}$\\$\cc{}~\cce{}~\ccd{}$\\$\nn{}~\nne{}~\nnd{}$\\$\zz{}~\zze{}~\zzd{}$\\$\qq{}~\uu{}~\kk{}~\pp{}$}}
\hfill
\begin{minipage}{.7\textwidth}
	\begin{lstlisting}[linewidth=\textwidth, language={[LaTeX]TeX}]
	$\rr{} \rre{} \rrd{} \rrm{} \rrme{} \rrp{} \rrpe{}$
	$\cc{} \cce{} \ccd{}$
	$\nn{} \nne{} \nnd{}$
	$\zz{} \zze{} \zzd{}$
	$\qq{} \uu{} \kk{} \pp{}$
	\end{lstlisting}
\end{minipage}

\setlength{\leftskip}{0pt}
\setlength{\textwidth}{18cm}%


%%
%% Uui
%%
\vspace*{.75cm}

\inCodeStub¡§fvtextcolor?bordeau!?\uui{!<n>§fvtextcolor?bordeau!?}!¡ -- Groupe des racines n-ième de l'unité.

\setlength{\leftskip}{.75cm}%
\setlength{\textwidth}{17.25cm}%

\colorbox{blue!15}{\pbox{.25\textwidth}{$\uui{6}$}}
\hfill
\begin{minipage}{.65\textwidth}
	\begin{lstlisting}[linewidth=\textwidth, language={[LaTeX]TeX}]
	$\uui{6}$
	\end{lstlisting}
\end{minipage}

\setlength{\leftskip}{0pt}
\setlength{\textwidth}{18cm}%


%%
%% Rectvect
%%
\vspace*{.75cm}

\inCodeStub¡§fvtextcolor?bordeau!?\rectvect{!<x>§fvtextcolor?bordeau!?}{!<y>§fvtextcolor?bordeau!?}!¡ -- Coordonnées d'un vecteur entre crochets et en colonne.

\setlength{\leftskip}{.75cm}%
\setlength{\textwidth}{17.25cm}%

\colorbox{blue!15}{\pbox{.25\textwidth}{$\rectvect{\dfrac{1}{2}}{2}$}}
\hfill
\begin{minipage}{.65\textwidth}
	\begin{lstlisting}[linewidth=\textwidth, language={[LaTeX]TeX}]
	$\rectvect{\dfrac{1}{2}}{2}$
	\end{lstlisting}
\end{minipage}

\setlength{\leftskip}{0pt}
\setlength{\textwidth}{18cm}%


%%
%% Aavect
%%
\vspace*{.75cm}

\inCodeStub¡§fvtextcolor?bordeau!?\aavect{!<x>§fvtextcolor?bordeau!?}{!<nom vecteur x>§fvtextcolor?bordeau!?}{!<y>§fvtextcolor?bordeau!?}{!<nom vecteur y>§fvtextcolor?bordeau!?}!¡ -- Coordonnées d'un vecteur entre parenthèses, en ligne et en notant les vecteurs de la base.

\setlength{\leftskip}{.75cm}%
\setlength{\textwidth}{17.25cm}%

\colorbox{blue!15}{\pbox{.25\textwidth}{$\aavect{\dfrac{1}{2}}{\imath}{2}{\jmath}$}}
\hfill
\begin{minipage}{.65\textwidth}
	\begin{lstlisting}[linewidth=\textwidth, language={[LaTeX]TeX}]
	$\aavect{\dfrac{1}{2}}{\imath}{2}{\jmath}$
	\end{lstlisting}
\end{minipage}

\setlength{\leftskip}{0pt}
\setlength{\textwidth}{18cm}%


%%
%% Avect
%%
\vspace*{.75cm}

\inCodeStub¡§fvtextcolor?bordeau!?\avect{!<x>§fvtextcolor?bordeau!?}{!<y>§fvtextcolor?bordeau!?}!¡ -- Coordonnées d'un vecteur entre parenthèses et en ligne.

\setlength{\leftskip}{.75cm}%
\setlength{\textwidth}{17.25cm}%

\colorbox{blue!15}{\pbox{.25\textwidth}{$\avect{\dfrac{1}{2}}{2}$}}
\hfill
\begin{minipage}{.65\textwidth}
	\begin{lstlisting}[linewidth=\textwidth, language={[LaTeX]TeX}]
	$\avect{\dfrac{1}{2}}{2}$
	\end{lstlisting}
\end{minipage}

\setlength{\leftskip}{0pt}
\setlength{\textwidth}{18cm}%


%%
%% Exercice
%%
\vspace*{.75cm}

\inCodeStub¡§fvtextcolor?bordeau!?\exercice{!<numéro exercice>§fvtextcolor?bordeau!?}{!<énoncé>§fvtextcolor?bordeau!?}!¡ -- Paragraphe pour introduire un exercice

\setlength{\leftskip}{.75cm}%
\setlength{\textwidth}{17.25cm}%

%\colorbox{blue!15}{\pbox{.35\textwidth}{\exercice{2}{Énoncé...}}}
\hfill
\begin{minipage}{.65\textwidth}
	\begin{lstlisting}[linewidth=\textwidth, language={[LaTeX]TeX}]
	\exercice{2}{Énoncé...}
	\end{lstlisting}
\end{minipage}

\setlength{\leftskip}{0pt}
\setlength{\textwidth}{18cm}%


%%
%% Exerciceds
%%
\vspace*{.75cm}

\inCodeStub¡§fvtextcolor?bordeau!?\exerciceds{!<numéro exercice>§fvtextcolor?bordeau!?}{!<énoncé>§fvtextcolor?bordeau!?}!¡ -- Paragraphe pour introduire un exercice dont l'énoncé est mis à la ligne

\setlength{\leftskip}{.75cm}%
\setlength{\textwidth}{17.25cm}%

%\colorbox{blue!15}{\pbox{.35\textwidth}{\exerciceds{2}{Énoncé...}}}
\hfill
\begin{minipage}{.65\textwidth}
	\begin{lstlisting}[linewidth=\textwidth, language={[LaTeX]TeX}]
	\exerciceds{2}{Énoncé...}
	\end{lstlisting}
\end{minipage}

\setlength{\leftskip}{0pt}
\setlength{\textwidth}{18cm}%


%%
%% norm
%%
\vspace*{.75cm}

\inCodeStub¡§fvtextcolor?bordeau!?\norm{!<vecteur>§fvtextcolor?bordeau!?}!¡ -- Barres de normes d'un vecteur.

\setlength{\leftskip}{.75cm}%
\setlength{\textwidth}{17.25cm}%

\colorbox{blue!15}{\pbox{.35\textwidth}{$\norm{\dfrac{1}{2}\vect{u}}$}}
\hfill
\begin{minipage}{.65\textwidth}
	\begin{lstlisting}[linewidth=\textwidth, language={[LaTeX]TeX}]
	$\norm{\dfrac{1}{2}\vect{u}}$
	\end{lstlisting}
\end{minipage}

\setlength{\leftskip}{0pt}
\setlength{\textwidth}{18cm}%


%%
%% Vnorm
%%
\vspace*{.75cm}

\inCodeStub¡§fvtextcolor?bordeau!?\vnorm{}!¡ -- Raccourci pour écrire la norme d'un vecteur $\vu$.

\setlength{\leftskip}{.75cm}%
\setlength{\textwidth}{17.25cm}%

\colorbox{blue!15}{\pbox{.35\textwidth}{$\vnorm{}$}}
\hfill
\begin{minipage}{.65\textwidth}
	\begin{lstlisting}[linewidth=\textwidth, language={[LaTeX]TeX}]
	$\vnorm{}$
	\end{lstlisting}
\end{minipage}

\setlength{\leftskip}{0pt}
\setlength{\textwidth}{18cm}%


%%
%% Prodscal
%%
\vspace*{.75cm}

\inCodeStub¡§fvtextcolor?bordeau!?\prodscal{!<premier vecteur>§fvtextcolor?bordeau!?}{!<second vecteur>§fvtextcolor?bordeau!?}!¡ -- Produit scalaire entre deux vecteurs, les noms sont fournis directement, car la commande place les flèches sur les noms de vecteurs.

\setlength{\leftskip}{.75cm}%
\setlength{\textwidth}{17.25cm}%

\colorbox{blue!15}{\pbox{.35\textwidth}{$\prodscal{u}{v}$}}
\hfill
\begin{minipage}{.65\textwidth}
	\begin{lstlisting}[linewidth=\textwidth, language={[LaTeX]TeX}]
	$\prodscal{u}{v}$
	\end{lstlisting}
\end{minipage}

\setlength{\leftskip}{0pt}
\setlength{\textwidth}{18cm}%


%%
%% Fctdisp
%%
\vspace*{.75cm}

\inCodeStub¡§fvtextcolor?bordeau!?\fctdisp{!<nom>§fvtextcolor?bordeau!?}{!<ensemble départ>§fvtextcolor?bordeau!?}{!<ensemble arrivée>§fvtextcolor?bordeau!?}{!<variable>§fvtextcolor?bordeau!?}{!<expression>§fvtextcolor?bordeau!?}!¡ -- Affichage d'une fonction sur deux lignes avec accolade.

\setlength{\leftskip}{.75cm}%
\setlength{\textwidth}{17.25cm}%

\colorbox{blue!15}{\pbox{.35\textwidth}{$\fctdisp{f}{\rre}{\rr}{x}{\dfrac{1}{x}}$}}
\hfill
\begin{minipage}{.65\textwidth}
	\begin{lstlisting}[linewidth=\textwidth, language={[LaTeX]TeX}]
	$\fctdisp{f}{\rre}{\rr}{x}{\dfrac{1}{x}}$
	\end{lstlisting}
\end{minipage}

\setlength{\leftskip}{0pt}
\setlength{\textwidth}{18cm}%


%%
%% Fctdispd
%%
\vspace*{.75cm}

\inCodeStub¡§fvtextcolor?bordeau!?\fctdispd{!<nom>§fvtextcolor?bordeau!?}{!<ensemble départ>§fvtextcolor?bordeau!?}{!<ensemble arrivée>§fvtextcolor?bordeau!?}{!<variable>§fvtextcolor?bordeau!?}{!<expression 1>§fvtextcolor?bordeau!?}!¡\newline
\phantom{i}~~~~~~~~~~~~~~~~~\inCodeStub¡§fvtextcolor?bordeau!?{!<variable>§fvtextcolor?bordeau!?}{!<expression 2>§fvtextcolor?bordeau!?}!¡ -- Affichage d'une fonction sur deux lignes avec accolade.

\setlength{\leftskip}{.75cm}%
\setlength{\textwidth}{17.25cm}%

\colorbox{blue!15}{\pbox{.35\textwidth}{$\fctdispd{f}{\rre}{\rr}{x}{\dfrac{1}{x}&\textnormal{~~si }x>0}{x}{-\dfrac{1}{x}&\textnormal{~~si }x\leq0}$}}
\hfill
\begin{minipage}{.65\textwidth}
	\begin{lstlisting}[linewidth=\textwidth, language={[LaTeX]TeX}]
	$\fctdispd{f}{\rre}{\rr}%
	{x}{\dfrac{1}{x} &\textnormal{~~si }x>0}%
	{x}{-\dfrac{1}{x} &\textnormal{~~si }x\leq0}$
	\end{lstlisting}
\end{minipage}

\setlength{\leftskip}{0pt}
\setlength{\textwidth}{18cm}%


%%
%% Fctdispsmall
%%
\vspace*{.75cm}

\inCodeStub¡§fvtextcolor?bordeau!?\fctdispsmall{!<nom>§fvtextcolor?bordeau!?}{!<ensemble de départ>§fvtextcolor?bordeau!?}{!<ensemble d'arrivée>§fvtextcolor?bordeau!?}!¡ -- Affichage simplifiée d'une fonction avec uniquement les ensembles de définition.

\setlength{\leftskip}{.75cm}%
\setlength{\textwidth}{17.25cm}%

\colorbox{blue!15}{\pbox{.35\textwidth}{$\fctdispsmall{f}{\rr}{\rre}$}}
\hfill
\begin{minipage}{.65\textwidth}
	\begin{lstlisting}[linewidth=\textwidth, language={[LaTeX]TeX}]
	$\fctdispsmall{f}{\rr}{\rre}$
	\end{lstlisting}
\end{minipage}

\setlength{\leftskip}{0pt}
\setlength{\textwidth}{18cm}%


%%
%% Fctdispsmallb
%%
\vspace*{.75cm}

\inCodeStub¡§fvtextcolor?bordeau!?\fctdispsmallb{!<nom>§fvtextcolor?bordeau!?}{!<ensemble de départ>§fvtextcolor?bordeau!?}{!<ensemble d'arrivée>§fvtextcolor?bordeau!?}!¡ -- Affichage simplifiée d'une fonction sous la forme d'une flèche liant les ensembles de définition et surmontée du nom de la fonction.

\setlength{\leftskip}{.75cm}%
\setlength{\textwidth}{17.25cm}%

\colorbox{blue!15}{\pbox{.35\textwidth}{$\fctdispsmallb{f}{\rr}{\rre}$}}
\hfill
\begin{minipage}{.65\textwidth}
	\begin{lstlisting}[linewidth=\textwidth, language={[LaTeX]TeX}]
	$\fctdispsmallb{f}{\rr}{\rre}$
	\end{lstlisting}
\end{minipage}

\setlength{\leftskip}{0pt}
\setlength{\textwidth}{18cm}%


%%
%% Fctdispsmallm
%%
\vspace*{.75cm}

\inCodeStub¡§fvtextcolor?bordeau!?\fctdispsmallm{!<nom>§fvtextcolor?bordeau!?}{!<variable>§fvtextcolor?bordeau!?}{!<expression>§fvtextcolor?bordeau!?}!¡ -- Affichage simplifiée d'une fonction avec uniquement la variable et l'expression.

\setlength{\leftskip}{.75cm}%
\setlength{\textwidth}{17.25cm}%

\colorbox{blue!15}{\pbox{.35\textwidth}{$\fctdispsmallm{f}{x}{\dfrac{1}{x}}$}}
\hfill
\begin{minipage}{.65\textwidth}
	\begin{lstlisting}[linewidth=\textwidth, language={[LaTeX]TeX}]
	$\fctdispsmallm{f}{x}{\dfrac{1}{x}}$
	\end{lstlisting}
\end{minipage}

\setlength{\leftskip}{0pt}
\setlength{\textwidth}{18cm}%


%%
%% Fctdispminimal
%%
\vspace*{.75cm}

\inCodeStub¡§fvtextcolor?bordeau!?\fctdispminimal{!<variable>§fvtextcolor?bordeau!?}{!<expression>§fvtextcolor?bordeau!?}!¡ -- Affichage simplifiée d'une fonction avec uniquement la variable et l'expression et sans le nom.

\setlength{\leftskip}{.75cm}%
\setlength{\textwidth}{17.25cm}%

\colorbox{blue!15}{\pbox{.35\textwidth}{$\fctdispminimal{x}{\dfrac{1}{x}}$}}
\hfill
\begin{minipage}{.65\textwidth}
	\begin{lstlisting}[linewidth=\textwidth, language={[LaTeX]TeX}]
	$\fctdispminimal{x}{\dfrac{1}{x}}$
	\end{lstlisting}
\end{minipage}

\setlength{\leftskip}{0pt}
\setlength{\textwidth}{18cm}%


%%
%% Curvenn
%%
\vspace*{.75cm}

\inCodeStub¡§fvtextcolor?bordeau!?\curvenn{!<nom courbe>§fvtextcolor?bordeau!?}{!<indice>§fvtextcolor?bordeau!?}!¡ -- Affichage d'un nom de courbe.

\setlength{\leftskip}{.75cm}%
\setlength{\textwidth}{17.25cm}%

\colorbox{blue!15}{\pbox{.35\textwidth}{$\curvenn{C}{1}$}}
\hfill
\begin{minipage}{.65\textwidth}
	\begin{lstlisting}[linewidth=\textwidth, language={[LaTeX]TeX}]
	$\curvenn{C}{1}$
	\end{lstlisting}
\end{minipage}

\setlength{\leftskip}{0pt}
\setlength{\textwidth}{18cm}%


%%
%% Tddet
%%
\vspace*{.75cm}

\inCodeStub¡§fvtextcolor?bordeau!?\tddet{!<valeur 1>§fvtextcolor?bordeau!?}{!<valeur 2>§fvtextcolor?bordeau!?}{!<valeur 3>§fvtextcolor?bordeau!?}{!<valeur 4>§fvtextcolor?bordeau!?}!¡ -- Déterminant de taille $2\times2$.

\setlength{\leftskip}{.75cm}%
\setlength{\textwidth}{17.25cm}%

\colorbox{blue!15}{\pbox{.35\textwidth}{$\tddet{1}{2}{3}{4}$}}
\hfill
\begin{minipage}{.65\textwidth}
	\begin{lstlisting}[linewidth=\textwidth, language={[LaTeX]TeX}]
	$\tddet{1}{2}{3}{4}$
	\end{lstlisting}
\end{minipage}

\setlength{\leftskip}{0pt}
\setlength{\textwidth}{18cm}%


%%
%% Overbar
%%
\vspace*{.75cm}

\inCodeStub¡§fvtextcolor?bordeau!?\overbar{!<valeur>§fvtextcolor?bordeau!?}!¡ -- Conjugué ou complémentaire d'une valeur.

\setlength{\leftskip}{.75cm}%
\setlength{\textwidth}{17.25cm}%

\colorbox{blue!15}{\pbox{.35\textwidth}{$\overbar{z_1-z_2}$}}
\hfill
\begin{minipage}{.65\textwidth}
	\begin{lstlisting}[linewidth=\textwidth, language={[LaTeX]TeX}]
	$\overbar{z_1-z_2}$
	\end{lstlisting}
\end{minipage}

\setlength{\leftskip}{0pt}
\setlength{\textwidth}{18cm}%


%%
%% Rep et Imp
%%
\vspace*{.75cm}

\inCodeStub¡§fvtextcolor?bordeau!?\rep{!<valeur>§fvtextcolor?bordeau!?}!, §fvtextcolor?bordeau!?\imp{!<valeur>§fvtextcolor?bordeau!?}!¡ -- Parties réelles et imaginaire d'un complexe.

\setlength{\leftskip}{.75cm}%
\setlength{\textwidth}{17.25cm}%

\colorbox{blue!15}{\pbox{.35\textwidth}{$z=\rep{z}+i\imp{z}$}}
\hfill
\begin{minipage}{.65\textwidth}
	\begin{lstlisting}[linewidth=\textwidth, language={[LaTeX]TeX}]
	$z=\rep{z}+i\imp{z}$
	\end{lstlisting}
\end{minipage}

\setlength{\leftskip}{0pt}
\setlength{\textwidth}{18cm}%


%%
%% Zarg
%%
\vspace*{.75cm}

\inCodeStub¡§fvtextcolor?bordeau!?\zarg{!<valeur>§fvtextcolor?bordeau!?}!¡ -- Argument d'un complexe.

\setlength{\leftskip}{.75cm}%
\setlength{\textwidth}{17.25cm}%

\colorbox{blue!15}{\pbox{.35\textwidth}{$\zarg{z}$}}
\hfill
\begin{minipage}{.65\textwidth}
	\begin{lstlisting}[linewidth=\textwidth, language={[LaTeX]TeX}]
	$\zarg{z}$
	\end{lstlisting}
\end{minipage}

\setlength{\leftskip}{0pt}
\setlength{\textwidth}{18cm}%


%%
%% Textenum
%%
\vspace*{.75cm}

\inCodeStub¡§fvtextcolor?bordeau!?\textenum{!<élément 1>§fvtextcolor?bordeau!?}{!<élément 2>§fvtextcolor?bordeau!?}{!<élément 3>§fvtextcolor?bordeau!?}!¡ -- Énumération d'éléments mathématiques dans un texte.

\setlength{\leftskip}{.75cm}%
\setlength{\textwidth}{17.25cm}%

\colorbox{blue!15}{\pbox{.35\textwidth}{Soient $\textenum{x}{y}{z}$...}}
\hfill
\begin{minipage}{.65\textwidth}
	\begin{lstlisting}[linewidth=\textwidth, language={[LaTeX]TeX}]
	Soient $\textenum{x}{y}{z}$...
	\end{lstlisting}
\end{minipage}

\setlength{\leftskip}{0pt}
\setlength{\textwidth}{18cm}%


%%
%% Textenumq
%%
\vspace*{.75cm}

\inCodeStub¡§fvtextcolor?bordeau!?\textenumq{!<élément 1>§fvtextcolor?bordeau!?}{!<élément 2>§fvtextcolor?bordeau!?}{!<élément 3>§fvtextcolor?bordeau!?}{!<élément 4>§fvtextcolor?bordeau!?}!¡ -- Énumération de quatre éléments mathématiques.

\setlength{\leftskip}{.75cm}%
\setlength{\textwidth}{17.25cm}%

\colorbox{blue!15}{\pbox{.35\textwidth}{Soient $\textenumq{x}{y}{z}{\alpha}$...}}
\hfill
\begin{minipage}{.65\textwidth}
	\begin{lstlisting}[linewidth=\textwidth, language={[LaTeX]TeX}]
	Soient $\textenumq{x}{y}{z}{\alpha}$...
	\end{lstlisting}
\end{minipage}

\setlength{\leftskip}{0pt}
\setlength{\textwidth}{18cm}%


%%
%% Dropsign
%%
\vspace*{.75cm}

\inCodeStub¡§fvtextcolor?bordeau!?\dropsign{!<symbole>§fvtextcolor?bordeau!?}!¡ -- Déporte un symbole en bas à gauche sous l'interligne.

\setlength{\leftskip}{.75cm}%
\setlength{\textwidth}{17.25cm}%

\colorbox{blue!15}{\hspace{.35cm}\pbox{.35\textwidth}{b$\dropsign{\times}$b\\Voici}}
\hfill
\begin{minipage}{.65\textwidth}
	\begin{lstlisting}[linewidth=\textwidth, language={[LaTeX]TeX}]
	b$\dropsign{\times}$b\\Voici
	\end{lstlisting}
\end{minipage}

Cette fonctionnalité est en partie utilisée pour écrire des divisions euclidiennes.

\colorbox{blue!15}{\pbox{.55\textwidth}{\hspace{0.4cm}$
  \begin{array}{r|r}
    \dropsign{-} 4x^4 - 12x^2 + 4x + 4 & 4x^3 - 6x + 1 \\ \cline{2-2}
    4x^4 + \phantom{1}6x^2 + \phantom{4}x \phantom{{}+4} & \phantom{4x^3 - 6}x\phantom{ {}+ 4}\\ \cline{1-1} \\[\dimexpr-\normalbaselineskip+\jot]
    \phantom{4x^4 +} 6x^2 + 3x + 4 \\
  \end{array}
$\rule{.1cm}{0pt}}}
\hfill
\begin{minipage}{.55\textwidth}
	\begin{lstlisting}[linewidth=\textwidth, language={[LaTeX]TeX}]
	\begin{array}{r|r}
		\dropsign{-} 4x^4 - 12x^2 + 4x + 4 & 4x^3 - 6x + 1 \\ \cline{2-2}
		4x^4 + \phantom{1}6x^2 + \phantom{4}x \phantom{{}+4} & \phantom{4x^3 - 6}x\phantom{ {}+ 4}\\ \cline{1-1} \\[\dimexpr-\normalbaselineskip+\jot]
		\phantom{4x^4 +} 6x^2 + 3x + 4 \\
	\end{array}
	\end{lstlisting}
\end{minipage}

\setlength{\leftskip}{0pt}
\setlength{\textwidth}{18cm}%


%%
%% Pge
%%
\vspace*{.75cm}

\inCodeStub¡§fvtextcolor?bordeau!?\pge{!<ensemble>§fvtextcolor?bordeau!?}!¡ -- Plus grand élément d'un ensemble.

\setlength{\leftskip}{.75cm}%
\setlength{\textwidth}{17.25cm}%

\colorbox{blue!15}{\pbox{.35\textwidth}{$\pge{E}$}}
\hfill
\begin{minipage}{.65\textwidth}
	\begin{lstlisting}[linewidth=\textwidth, language={[LaTeX]TeX}]
	$\pge{E}$
	\end{lstlisting}
\end{minipage}

\setlength{\leftskip}{0pt}
\setlength{\textwidth}{18cm}%


%%
%% Pgcd
%%
\vspace*{.75cm}

\inCodeStub¡§fvtextcolor?bordeau!?\pgcd{!<valeur 1>§fvtextcolor?bordeau!?}{!<valeur 2>§fvtextcolor?bordeau!?}!¡ -- PGCD de deux valeurs

\setlength{\leftskip}{.75cm}%
\setlength{\textwidth}{17.25cm}%

\colorbox{blue!15}{\pbox{.35\textwidth}{$\pgcd{a_1}{a_2}$}}
\hfill
\begin{minipage}{.65\textwidth}
	\begin{lstlisting}[linewidth=\textwidth, language={[LaTeX]TeX}]
	$\pgcd{a_1}{a_2}$
	\end{lstlisting}
\end{minipage}

\setlength{\leftskip}{0pt}
\setlength{\textwidth}{18cm}%


%%
%% Ppcm
%%
\vspace*{.75cm}

\inCodeStub¡§fvtextcolor?bordeau!?\ppcm{!<valeur 1>§fvtextcolor?bordeau!?}{!<valeur 2>§fvtextcolor?bordeau!?}!¡ -- PPCM de deux valeurs

\setlength{\leftskip}{.75cm}%
\setlength{\textwidth}{17.25cm}%

\colorbox{blue!15}{\pbox{.35\textwidth}{$\ppcm{a_1}{a_2}$}}
\hfill
\begin{minipage}{.65\textwidth}
	\begin{lstlisting}[linewidth=\textwidth, language={[LaTeX]TeX}]
	$\ppcm{a_1}{a_2}$
	\end{lstlisting}
\end{minipage}

\setlength{\leftskip}{0pt}
\setlength{\textwidth}{18cm}%


%%
%% Comp
%%
\vspace*{.75cm}

\inCodeStub¡§fvtextcolor?bordeau!?\comp{!<fonction 1>§fvtextcolor?bordeau!?}{!<fonction 2>§fvtextcolor?bordeau!?}!¡ -- Composée de deux fonctions.

\setlength{\leftskip}{.75cm}%
\setlength{\textwidth}{17.25cm}%

\colorbox{blue!15}{\pbox{.35\textwidth}{$\comp{f_1}{f_2}$}}
\hfill
\begin{minipage}{.65\textwidth}
	\begin{lstlisting}[linewidth=\textwidth, language={[LaTeX]TeX}]
	$\comp{f_1}{f_2}$
	\end{lstlisting}
\end{minipage}

\setlength{\leftskip}{0pt}
\setlength{\textwidth}{18cm}%


%%
%% Compt
%%
\vspace*{.75cm}

\inCodeStub¡§fvtextcolor?bordeau!?\compt{!<fonction 1>§fvtextcolor?bordeau!?}{!<fonction 2>§fvtextcolor?bordeau!?}{!<fonction 3>§fvtextcolor?bordeau!?}!¡ -- Raccourci pour la composée de trois fonctions.

\setlength{\leftskip}{.75cm}%
\setlength{\textwidth}{17.25cm}%

\colorbox{blue!15}{\pbox{.35\textwidth}{$\compt{f_1}{f_2}{f_3}$}}
\hfill
\begin{minipage}{.65\textwidth}
	\begin{lstlisting}[linewidth=\textwidth, language={[LaTeX]TeX}]
	$\compt{f_1}{f_2}{f_3}$
	\end{lstlisting}
\end{minipage}

\setlength{\leftskip}{0pt}
\setlength{\textwidth}{18cm}%


%%
%% Comptt
%%
\vspace*{.75cm}

\inCodeStub¡§fvtextcolor?bordeau!?\comptt{!<fonction>§fvtextcolor?bordeau!?}{!<fonction encadrante>§fvtextcolor?bordeau!?}!¡ -- Fonction composée avec une fonction et son inverse.

\setlength{\leftskip}{.75cm}%
\setlength{\textwidth}{17.25cm}%

\colorbox{blue!15}{\pbox{.35\textwidth}{$\comptt{f}{\varphi}$}}
\hfill
\begin{minipage}{.65\textwidth}
	\begin{lstlisting}[linewidth=\textwidth, language={[LaTeX]TeX}]
	$\comptt{f}{\varphi}$
	\end{lstlisting}
\end{minipage}

\setlength{\leftskip}{0pt}
\setlength{\textwidth}{18cm}%


%%
%% Fctid
%%
\vspace*{.75cm}

\inCodeStub¡§fvtextcolor?bordeau!?\fctid{!<ensemble>§fvtextcolor?bordeau!?}!¡ -- Fonction identité sur un ensemble.

\setlength{\leftskip}{.75cm}%
\setlength{\textwidth}{17.25cm}%

\colorbox{blue!15}{\pbox{.35\textwidth}{$\fctid{E}$}}
\hfill
\begin{minipage}{.65\textwidth}
	\begin{lstlisting}[linewidth=\textwidth, language={[LaTeX]TeX}]
	$\fctid{E}$
	\end{lstlisting}
\end{minipage}

\setlength{\leftskip}{0pt}
\setlength{\textwidth}{18cm}%


%%
%% Proved
%%
\vspace*{.75cm}

\inCodeStub¡§fvtextcolor?bordeau!?\proved{}!¡ -- Carré noire pour terminer une preuve. S'aligne à droite en fin de ligne.

\setlength{\leftskip}{.75cm}%
\setlength{\textwidth}{17.25cm}%

\colorbox{blue!15}{\begin{minipage}{.30\textwidth}CQFD.\proved{}\end{minipage}}
\hfill
\begin{minipage}{.65\textwidth}
	\begin{lstlisting}[linewidth=\textwidth, language={[LaTeX]TeX}]
	CQFD.\proved{}
	\end{lstlisting}
\end{minipage}

\setlength{\leftskip}{0pt}
\setlength{\textwidth}{18cm}%


%%
%% Equationlabel
%%
\vspace*{.75cm}

\inCodeStub¡§fvtextcolor?bordeau!?\equationlabel{!<équation>§fvtextcolor?bordeau!?}{!<label>§fvtextcolor?bordeau!?}{!<numéro>§fvtextcolor?bordeau!?}!¡ -- Raccourci pour afficher une équation et lui associer un nom ainsi qu'un label pour références futures.

\setlength{\leftskip}{.75cm}%
\setlength{\textwidth}{17.25cm}%

\colorbox{blue!15}{\begin{minipage}{.30\textwidth}\equationlabel{a=b}{moneqn}{1}\newline
Équation~\ref{moneqn}\end{minipage}}
\hfill
\begin{minipage}{.65\textwidth}
	\begin{lstlisting}[linewidth=\textwidth, language={[LaTeX]TeX}]
	\equationlabel{a=b}{moneqn}{1}
	Équation~\ref{moneqn}
	\end{lstlisting}
\end{minipage}

\setlength{\leftskip}{0pt}
\setlength{\textwidth}{18cm}%


%%
%% Questlabel
%%
\vspace*{.75cm}

\inCodeStub¡§fvtextcolor?bordeau!?\questlabel{!<numéro>§fvtextcolor?bordeau!?}!¡ -- Numérotation d'une question en accord avec le style des exercices.

\setlength{\leftskip}{.75cm}%
\setlength{\textwidth}{17.25cm}%

\colorbox{blue!15}{\pbox{.35\textwidth}{\questlabel{1.a}}}
\hfill
\begin{minipage}{.65\textwidth}
	\begin{lstlisting}[linewidth=\textwidth, language={[LaTeX]TeX}]
	\questlabel{1.a}
	\end{lstlisting}
\end{minipage}

\setlength{\leftskip}{0pt}
\setlength{\textwidth}{18cm}%


%%
%% Perm
%%
\vspace*{.75cm}

\inCodeStub¡§fvtextcolor?bordeau!?\perm{!<nombre permutations>§fvtextcolor?bordeau!?}!¡ -- Groupe des permutations

\setlength{\leftskip}{.75cm}%
\setlength{\textwidth}{17.25cm}%

\colorbox{blue!15}{\pbox{.35\textwidth}{$\perm{3}$}}
\hfill
\begin{minipage}{.65\textwidth}
	\begin{lstlisting}[linewidth=\textwidth, language={[LaTeX]TeX}]
	$\perm{3}$
	\end{lstlisting}
\end{minipage}

\setlength{\leftskip}{0pt}
\setlength{\textwidth}{18cm}%


%%
%% Conj
%%
\vspace*{.75cm}

\inCodeStub¡§fvtextcolor?bordeau!?\conj{!<fonction 1>§fvtextcolor?bordeau!?}{!<fonction 2>§fvtextcolor?bordeau!?}!¡ -- Conjugaison de deux fonctions.

\setlength{\leftskip}{.75cm}%
\setlength{\textwidth}{17.25cm}%

\colorbox{blue!15}{\pbox{.35\textwidth}{$\conj{f}{g}$}}
\hfill
\begin{minipage}{.65\textwidth}
	\begin{lstlisting}[linewidth=\textwidth, language={[LaTeX]TeX}]
	$\conj{f}{g}$
	\end{lstlisting}
\end{minipage}

\setlength{\leftskip}{0pt}
\setlength{\textwidth}{18cm}%


%%
%% Parent
%%
\vspace*{.75cm}

\inCodeStub¡§fvtextcolor?bordeau!?\parent{!<expression>§fvtextcolor?bordeau!?}!¡ -- Mettre une expression entre parenthèses.

\setlength{\leftskip}{.75cm}%
\setlength{\textwidth}{17.25cm}%

\colorbox{blue!15}{\pbox{.35\textwidth}{$\parent{\dfrac{1}{2}}$}}
\hfill
\begin{minipage}{.65\textwidth}
	\begin{lstlisting}[linewidth=\textwidth, language={[LaTeX]TeX}]
	$\parent{\dfrac{1}{2}}$
	\end{lstlisting}
\end{minipage}

\setlength{\leftskip}{0pt}
\setlength{\textwidth}{18cm}%


%%
%% Intcroch
%%
\vspace*{.75cm}

\inCodeStub¡§fvtextcolor?bordeau!?\intcroch{!<expression>§fvtextcolor?bordeau!?}{!<borne inférieure>§fvtextcolor?bordeau!?}{!<borne supérieure>§fvtextcolor?bordeau!?}!¡ -- Écriture en crochet pour évaluer la différence d'une expression entre sa borne supérieure et sa borne inférieure lors d'un calcul intégral.

\setlength{\leftskip}{.75cm}%
\setlength{\textwidth}{17.25cm}%

\colorbox{blue!15}{\pbox{.35\textwidth}{$\intcroch{f(x)}{0}{2}$}}
\hfill
\begin{minipage}{.65\textwidth}
	\begin{lstlisting}[linewidth=\textwidth, language={[LaTeX]TeX}]
	$\intcroch{f(x)}{0}{2}$
	\end{lstlisting}
\end{minipage}

\setlength{\leftskip}{0pt}
\setlength{\textwidth}{18cm}%


%%
%% Zconj
%%
\vspace*{.75cm}

\inCodeStub¡§fvtextcolor?bordeau!?\zconj{!<complexe>§fvtextcolor?bordeau!?}!¡ -- Conjugué d'un nombre complexe.

\setlength{\leftskip}{.75cm}%
\setlength{\textwidth}{17.25cm}%

\colorbox{blue!15}{\pbox{.35\textwidth}{$\zconj{z_1}$}}
\hfill
\begin{minipage}{.65\textwidth}
	\begin{lstlisting}[linewidth=\textwidth, language={[LaTeX]TeX}]
	$\zconj{z_1}$
	\end{lstlisting}
\end{minipage}

\setlength{\leftskip}{0pt}
\setlength{\textwidth}{18cm}%


%%
%% Congr
%%
\vspace*{.75cm}

\inCodeStub¡§fvtextcolor?bordeau!?\congr{!<classe de congruence>§fvtextcolor?bordeau!?}!¡ -- Mise en forme de la classe de congruence.

\setlength{\leftskip}{.75cm}%
\setlength{\textwidth}{17.25cm}%

\colorbox{blue!15}{\pbox{.35\textwidth}{une expression$\congr{2\pi}$}}
\hfill
\begin{minipage}{.65\textwidth}
	\begin{lstlisting}[linewidth=\textwidth, language={[LaTeX]TeX}]
	une expression$\congr{2\pi}$
	\end{lstlisting}
\end{minipage}

\setlength{\leftskip}{0pt}
\setlength{\textwidth}{18cm}%


%%
%% Congru
%%
\vspace*{.75cm}

\inCodeStub¡§fvtextcolor?bordeau!?\congr{!<valeur 1>§fvtextcolor?bordeau!?}{!<valeur 2>§fvtextcolor?bordeau!?}{!<classe de congruence>§fvtextcolor?bordeau!?}!¡ -- Écriture d'une congruence.

\setlength{\leftskip}{.75cm}%
\setlength{\textwidth}{17.25cm}%

\colorbox{blue!15}{\pbox{.35\textwidth}{$\congru{a}{b}{2\pi}$}}
\hfill
\begin{minipage}{.65\textwidth}
	\begin{lstlisting}[linewidth=\textwidth, language={[LaTeX]TeX}]
	$\congru{a}{b}{2\pi}$
	\end{lstlisting}
\end{minipage}

\setlength{\leftskip}{0pt}
\setlength{\textwidth}{18cm}%


%%
%% Ncongru
%%
\vspace*{.75cm}

\inCodeStub¡§fvtextcolor?bordeau!?\ncongr{!<valeur 1>§fvtextcolor?bordeau!?}{!<valeur 2>§fvtextcolor?bordeau!?}{!<classe de congruence>§fvtextcolor?bordeau!?}!¡ -- Écriture d'une non congruence.

\setlength{\leftskip}{.75cm}%
\setlength{\textwidth}{17.25cm}%

\colorbox{blue!15}{\pbox{.35\textwidth}{$\ncongru{a}{b}{2\pi}$}}
\hfill
\begin{minipage}{.65\textwidth}
	\begin{lstlisting}[linewidth=\textwidth, language={[LaTeX]TeX}]
	$\ncongru{a}{b}{2\pi}$
	\end{lstlisting}
\end{minipage}

\setlength{\leftskip}{0pt}
\setlength{\textwidth}{18cm}%


%%
%% Card
%%
\vspace*{.75cm}

\inCodeStub¡§fvtextcolor?bordeau!?\card{!<ensemble>§fvtextcolor?bordeau!?}!¡ -- Cardinal d'un ensemble.

\setlength{\leftskip}{.75cm}%
\setlength{\textwidth}{17.25cm}%

\colorbox{blue!15}{\pbox{.35\textwidth}{$\card{E}$}}
\hfill
\begin{minipage}{.65\textwidth}
	\begin{lstlisting}[linewidth=\textwidth, language={[LaTeX]TeX}]
	$\card{E}$
	\end{lstlisting}
\end{minipage}

\setlength{\leftskip}{0pt}
\setlength{\textwidth}{18cm}%


%%
%% Dist
%%
\vspace*{.75cm}

\inCodeStub¡§fvtextcolor?bordeau!?\dist{!<point 1>§fvtextcolor?bordeau!?}{!<point 2>§fvtextcolor?bordeau!?}!¡ -- Distance entre deux points.

\setlength{\leftskip}{.75cm}%
\setlength{\textwidth}{17.25cm}%

\colorbox{blue!15}{\pbox{.35\textwidth}{$\dist{A}{B}$}}
\hfill
\begin{minipage}{.65\textwidth}
	\begin{lstlisting}[linewidth=\textwidth, language={[LaTeX]TeX}]
	$\dist{A}{B}$
	\end{lstlisting}
\end{minipage}

\setlength{\leftskip}{0pt}
\setlength{\textwidth}{18cm}%


%%
%% Entint
%%
\vspace*{.75cm}

\inCodeStub¡§fvtextcolor?bordeau!?\entint{!<borne inférieure>§fvtextcolor?bordeau!?}{!<borne supérieure>§fvtextcolor?bordeau!?}!¡ -- Intervalle fermé d'entiers.

\setlength{\leftskip}{.75cm}%
\setlength{\textwidth}{17.25cm}%

\colorbox{blue!15}{\pbox{.35\textwidth}{$\entint{1}{n}$}}
\hfill
\begin{minipage}{.65\textwidth}
	\begin{lstlisting}[linewidth=\textwidth, language={[LaTeX]TeX}]
	$\entint{1}{n}$
	\end{lstlisting}
\end{minipage}

\setlength{\leftskip}{0pt}
\setlength{\textwidth}{18cm}%


%%
%% Intint
%%
\vspace*{.75cm}

\inCodeStub¡§fvtextcolor?bordeau!?\intint{!<délimiteur inf>§fvtextcolor?bordeau!?}{!<borne inf>§fvtextcolor?bordeau!?}{!<borne sup>§fvtextcolor?bordeau!?}{!<délimiteur sup>§fvtextcolor?bordeau!?}!¡ -- Intervalle s'adaptant à la taille de son contenu pour lequel les délimiteurs et les valeurs limites sont à préciser.

\setlength{\leftskip}{.75cm}%
\setlength{\textwidth}{17.25cm}%

\colorbox{blue!15}{\pbox{.35\textwidth}{$\intint{]}{1}{\dfrac{3}{2}}{]}$}}
\hfill
\begin{minipage}{.65\textwidth}
	\begin{lstlisting}[linewidth=\textwidth, language={[LaTeX]TeX}]
	$\intint{]}{1}{\dfrac{3}{2}}{]}$
	\end{lstlisting}
\end{minipage}

\setlength{\leftskip}{0pt}
\setlength{\textwidth}{18cm}%


%%
%% Closeint
%%
\vspace*{.75cm}

\inCodeStub¡§fvtextcolor?bordeau!?\closeint{!<borne inférieur>§fvtextcolor?bordeau!?}{!<borne supérieure>§fvtextcolor?bordeau!?}!¡ -- Intervalle fermé s'adaptant à son contenu.

\setlength{\leftskip}{.75cm}%
\setlength{\textwidth}{17.25cm}%

\colorbox{blue!15}{\pbox{.35\textwidth}{$\closeint{1}{\dfrac{3}{2}}$}}
\hfill
\begin{minipage}{.65\textwidth}
	\begin{lstlisting}[linewidth=\textwidth, language={[LaTeX]TeX}]
	$\closeint{1}{\dfrac{3}{2}}$
	\end{lstlisting}
\end{minipage}

\setlength{\leftskip}{0pt}
\setlength{\textwidth}{18cm}%


%%
%% Tq
%%
\vspace*{.75cm}

\inCodeStub¡§fvtextcolor?bordeau!?\tq{}!¡ -- Affichage de la barre signifiant \og tel que\fg{} dans la description d'un ensemble par exemple.

\setlength{\leftskip}{.75cm}%
\setlength{\textwidth}{17.25cm}%

\colorbox{blue!15}{\pbox{.35\textwidth}{$\{x\in E \tq{} x\leq0\}$}}
\hfill
\begin{minipage}{.65\textwidth}
	\begin{lstlisting}[linewidth=\textwidth, language={[LaTeX]TeX}]
	$\{x\in E \tq{} x\leq0\}$
	\end{lstlisting}
\end{minipage}

\setlength{\leftskip}{0pt}
\setlength{\textwidth}{18cm}%


%%
%% Ddfrac
%%
\vspace*{.75cm}

\inCodeStub¡§fvtextcolor?bordeau!?\ddfrac{!<numérateur>§fvtextcolor?bordeau!?}{!<dénominateur>§fvtextcolor?bordeau!?}!¡ -- Extension de \texttt{dfrac} n'appliquant pas de réductions de police au numérateur et au dénominateur. À utiliser avec parcimonie, car peut rendre la lecture plus difficile.

\setlength{\leftskip}{.75cm}%
\setlength{\textwidth}{17.25cm}%

\colorbox{blue!15}{\pbox{.35\textwidth}{$\ddfrac{\frac{1+x}{y}}{\frac{y}{x}}$}}
\hfill
\begin{minipage}{.65\textwidth}
	\begin{lstlisting}[linewidth=\textwidth, language={[LaTeX]TeX}]
	$\ddfrac{\frac{1+x}{y}}{\frac{y}{x}}$
	\end{lstlisting}
\end{minipage}

\setlength{\leftskip}{0pt}
\setlength{\textwidth}{18cm}%


%%
%% Diffdchar, Dd
%%
\vspace*{.75cm}

\inCodeStub¡§fvtextcolor?bordeau!?\diffdchar{}!, §fvtextcolor?bordeau!?\dd{}!¡ -- Caractère différentiel droit pour les dérivées.

\setlength{\leftskip}{.75cm}%
\setlength{\textwidth}{17.25cm}%

\colorbox{blue!15}{\pbox{.35\textwidth}{$\diffdchar{}$~$\dd{}$}}
\hfill
\begin{minipage}{.65\textwidth}
	\begin{lstlisting}[linewidth=\textwidth, language={[LaTeX]TeX}]
	$\diffdchar{}$~$\dd{}$
	\end{lstlisting}
\end{minipage}

\setlength{\leftskip}{0pt}
\setlength{\textwidth}{18cm}%


%%
%% der
%%
\vspace*{.75cm}

\inCodeStub¡§fvtextcolor?bordeau!?\der{!<variable>§fvtextcolor?bordeau!?}!¡ -- Dérivée par rapport à la variable.

\setlength{\leftskip}{.75cm}%
\setlength{\textwidth}{17.25cm}%

\colorbox{blue!15}{\pbox{.35\textwidth}{$\der{x}$}}
\hfill
\begin{minipage}{.65\textwidth}
	\begin{lstlisting}[linewidth=\textwidth, language={[LaTeX]TeX}]
	$\der{x}$
	\end{lstlisting}
\end{minipage}

\setlength{\leftskip}{0pt}
\setlength{\textwidth}{18cm}%


%%
%% dder
%%
\vspace*{.75cm}

\inCodeStub¡§fvtextcolor?bordeau!?\dder{!<fonction>§fvtextcolor?bordeau!?}{!<variable>§fvtextcolor?bordeau!?}!¡ -- Dérivée d'une fonction par rapport à la variable.

\setlength{\leftskip}{.75cm}%
\setlength{\textwidth}{17.25cm}%

\colorbox{blue!15}{\pbox{.35\textwidth}{$\dder{f}{x}$}}
\hfill
\begin{minipage}{.65\textwidth}
	\begin{lstlisting}[linewidth=\textwidth, language={[LaTeX]TeX}]
	$\dder{f}{x}$
	\end{lstlisting}
\end{minipage}

\setlength{\leftskip}{0pt}
\setlength{\textwidth}{18cm}%


%%
%% dderp
%%
\vspace*{.75cm}

\inCodeStub¡§fvtextcolor?bordeau!?\dderp{!<fonction>§fvtextcolor?bordeau!?}{!<variable>§fvtextcolor?bordeau!?}{!<ordre>§fvtextcolor?bordeau!?}!¡ -- Dérivée d'une fonction par rapport à la variable à un ordre donné.

\setlength{\leftskip}{.75cm}%
\setlength{\textwidth}{17.25cm}%

\colorbox{blue!15}{\pbox{.35\textwidth}{$\dderp{f}{x}{2}$}}
\hfill
\begin{minipage}{.65\textwidth}
	\begin{lstlisting}[linewidth=\textwidth, language={[LaTeX]TeX}]
	$\dderp{f}{x}{2}$
	\end{lstlisting}
\end{minipage}

\setlength{\leftskip}{0pt}
\setlength{\textwidth}{18cm}%


%%
%% Integrale
%%
\vspace*{.75cm}

\inCodeStub¡§fvtextcolor?bordeau!?\integrale{!<bornes et expression>§fvtextcolor?bordeau!?}{!<variable>§fvtextcolor?bordeau!?}!¡ -- Intégrale d'une fonction données, en précisant ou non les bornes, et en fournissant la variable par rapport à laquelle l'intégration est réalisée.

\setlength{\leftskip}{.75cm}%
\setlength{\textwidth}{17.25cm}%

\colorbox{blue!15}{\pbox{.35\textwidth}{$\integrale{_0^1f(x)}{x}$}}
\hfill
\begin{minipage}{.65\textwidth}
	\begin{lstlisting}[linewidth=\textwidth, language={[LaTeX]TeX}]
	$\integrale{_0^1f(x)}{x}$
	\end{lstlisting}
\end{minipage}

\setlength{\leftskip}{0pt}
\setlength{\textwidth}{18cm}%


%%
%% Ecrbase
%%
\vspace*{.75cm}

\inCodeStub¡§fvtextcolor?bordeau!?\ecrbase{!<nombre>§fvtextcolor?bordeau!?}{!<base>§fvtextcolor?bordeau!?}!¡ -- Écriture d'un nombre dans une base donnée.

\setlength{\leftskip}{.75cm}%
\setlength{\textwidth}{17.25cm}%

\colorbox{blue!15}{\pbox{.35\textwidth}{$\ecrbase{1001010}{2}$}}
\hfill
\begin{minipage}{.65\textwidth}
	\begin{lstlisting}[linewidth=\textwidth, language={[LaTeX]TeX}]
	$\ecrbase{1001010}{2}$
	\end{lstlisting}
\end{minipage}

\setlength{\leftskip}{0pt}
\setlength{\textwidth}{18cm}%


%%
%% Fctname
%%
\vspace*{.75cm}

\inCodeStub¡§fvtextcolor?bordeau!?\fctname{!<lettre majuscule>§fvtextcolor?bordeau!?}!¡ -- À définir.

\setlength{\leftskip}{.75cm}%
\setlength{\textwidth}{17.25cm}%

\colorbox{blue!15}{\pbox{.35\textwidth}{$\fctname{A}$}}
\hfill
\begin{minipage}{.65\textwidth}
	\begin{lstlisting}[linewidth=\textwidth, language={[LaTeX]TeX}]
	$\fctname{A}$
	\end{lstlisting}
\end{minipage}

\setlength{\leftskip}{0pt}
\setlength{\textwidth}{18cm}%


%%
%% Angps
%%
\vspace*{.75cm}

\inCodeStub¡§fvtextcolor?bordeau!?\angps{!<vecteur 1>§fvtextcolor?bordeau!?}{!<vecteur 2>§fvtextcolor?bordeau!?}!¡ -- Produit scalaire entre deux vecteur, notation non pointée.

\setlength{\leftskip}{.75cm}%
\setlength{\textwidth}{17.25cm}%

\colorbox{blue!15}{\pbox{.35\textwidth}{$\angps{\vect{u}}{\vect{v}}$}}
\hfill
\begin{minipage}{.65\textwidth}
	\begin{lstlisting}[linewidth=\textwidth, language={[LaTeX]TeX}]
	$\angps{\vect{u}}{\vect{v}}$
	\end{lstlisting}
\end{minipage}

\setlength{\leftskip}{0pt}
\setlength{\textwidth}{18cm}%


%%
%% Transp
%%
\vspace*{.75cm}

\inCodeStub¡§fvtextcolor?bordeau!?\transp{}!¡ -- Notation de la transposition.

\setlength{\leftskip}{.75cm}%
\setlength{\textwidth}{17.25cm}%

\colorbox{blue!15}{\pbox{.35\textwidth}{$\transp{}M$, $M\transp{}$}}
\hfill
\begin{minipage}{.65\textwidth}
	\begin{lstlisting}[linewidth=\textwidth, language={[LaTeX]TeX}]
	$\transp{}M$, $M\transp{}$
	\end{lstlisting}
\end{minipage}

\setlength{\leftskip}{0pt}
\setlength{\textwidth}{18cm}%


%%
%% Entpartname
%%
\vspace*{.75cm}

\inCodeStub¡§fvtextcolor?bordeau!?\entpartname{}!¡ -- Fonction partie entière.

\setlength{\leftskip}{.75cm}%
\setlength{\textwidth}{17.25cm}%

\colorbox{blue!15}{\pbox{.35\textwidth}{$\entpartname{}(3.5)$}}
\hfill
\begin{minipage}{.65\textwidth}
	\begin{lstlisting}[linewidth=\textwidth, language={[LaTeX]TeX}]
	$\entpartname{}(3.5)$
	\end{lstlisting}
\end{minipage}

\setlength{\leftskip}{0pt}
\setlength{\textwidth}{18cm}%


%%
%% Entpart
%%
\vspace*{.75cm}

\inCodeStub¡§fvtextcolor?bordeau!?\entpart{}!¡ -- Fonction partie entière appliquée à une valeur.

\setlength{\leftskip}{.75cm}%
\setlength{\textwidth}{17.25cm}%

\colorbox{blue!15}{\pbox{.35\textwidth}{$\entpart{\dfrac{3}{2}}$}}
\hfill
\begin{minipage}{.65\textwidth}
	\begin{lstlisting}[linewidth=\textwidth, language={[LaTeX]TeX}]
	$\entpart{\dfrac{3}{2}}$
	\end{lstlisting}
\end{minipage}

\setlength{\leftskip}{0pt}
\setlength{\textwidth}{18cm}%


%%
%% Complexitealg
%%
\vspace*{.75cm}

\inCodeStub¡§fvtextcolor?bordeau!?\complexitealg{!<complexité>§fvtextcolor?bordeau!?}!¡ -- Écriture de la complexité algorithmique.

\setlength{\leftskip}{.75cm}%
\setlength{\textwidth}{17.25cm}%

\colorbox{blue!15}{\pbox{.35\textwidth}{$\complexitealg{n^2}$}}
\hfill
\begin{minipage}{.65\textwidth}
	\begin{lstlisting}[linewidth=\textwidth, language={[LaTeX]TeX}]
	$\complexitealg{n^2}$
	\end{lstlisting}
\end{minipage}

\setlength{\leftskip}{0pt}
\setlength{\textwidth}{18cm}%


%%
%% Complexitealgsub
%%
\vspace*{.75cm}

\inCodeStub¡§fvtextcolor?bordeau!?\complexitealgsub{!<complexité>§fvtextcolor?bordeau!?}{!<détail>§fvtextcolor?bordeau!?}!¡ -- Écriture de la complexité algorithmique avec précision sur l'algorithme.

\setlength{\leftskip}{.75cm}%
\setlength{\textwidth}{17.25cm}%

\colorbox{blue!15}{\pbox{.35\textwidth}{$\complexitealgsub{n\ln(n)}{quick sort}$}}
\hfill
\begin{minipage}{.65\textwidth}
	\begin{lstlisting}[linewidth=\textwidth, language={[LaTeX]TeX}]
	$\complexitealgsub{n\ln(n)}{quick sort}$
	\end{lstlisting}
\end{minipage}

\setlength{\leftskip}{0pt}
\setlength{\textwidth}{18cm}%


%%
%% Eqna
%%
\vspace*{.75cm}

\inCodeStub¡§fvtextcolor?bordeau!?\eqna{!<système>§fvtextcolor?bordeau!?}!¡ -- Alignement d'équations, sans accolades. Les informations d'alignement sont contenues dans le paramètre fourni. Il n'est pas nécessaire de détailler les différentes colonnes possibles.

\setlength{\leftskip}{.75cm}%
\setlength{\textwidth}{17.25cm}%

\colorbox{blue!15}{\pbox{.35\textwidth}{$\eqna{
	\parent{\ch(x)}^3&=\parent{\dfrac{e^x+e^{-x}}{2}}^3\\
	                 &=...\\
	                 &=\frac{1}{4}\ch(3x)+\frac{3}{4}\ch(x)
}$}}
\hfill
\begin{minipage}{.65\textwidth}
	\begin{lstlisting}[linewidth=\textwidth, language={[LaTeX]TeX}]
	$\eqna{
	\parent{\ch(x)}^3&=\parent{\dfrac{e^x+e^{-x}}{2}}^3\\
	                 &=...\\
	                 &=\frac{1}{4}\ch(3x)+\frac{3}{4}\ch(x)
	}$
	\end{lstlisting}
\end{minipage}

\setlength{\leftskip}{0pt}
\setlength{\textwidth}{18cm}%


%%
%% Syst
%%
\vspace*{.75cm}

\inCodeStub¡§fvtextcolor?bordeau!?\syst{!<système>§fvtextcolor?bordeau!?}!¡ -- Alignement d'équations, avec accolades. Les informations d'alignement sont contenues dans le paramètre fourni. Il n'est pas nécessaire de détailler les différentes colonnes possibles.

\setlength{\leftskip}{.75cm}%
\setlength{\textwidth}{17.25cm}%

\colorbox{blue!15}{\pbox{.35\textwidth}{$y=\argch(x)\mathequiv{.25cm}\syst{x&=\ch(y)\\y&\geq0}$}}
\hfill
\begin{minipage}{.65\textwidth}
	\begin{lstlisting}[linewidth=\textwidth, language={[LaTeX]TeX}]
	$y=\argch(x)\mathequiv{.25cm}\syst{x&=\ch(y)\\y&\geq0}$
	\end{lstlisting}
\end{minipage}

\setlength{\leftskip}{0pt}
\setlength{\textwidth}{18cm}%


%%
%% Systa
%%
\vspace*{.75cm}

\inCodeStub¡§fvtextcolor?bordeau!?\systa{!<en têtes colonnes>§fvtextcolor?bordeau!?}{!<système>§fvtextcolor?bordeau!?}!¡ -- Alignement d'équations, avec accolades et colonnes spécifiques pour ranger les différents termes. Il s'agit de la forme la plus contraignante de systèmes proposée, à utiliser dans le cas où l'alignement des termes à une utilité forte (pivot de Gauss par exemple).

\setlength{\leftskip}{.75cm}%
\setlength{\textwidth}{17.25cm}%

\colorbox{blue!15}{\pbox{.35\textwidth}{$\systa{rororor}{%
	x  & + &ay & - & z  & = & \beta \\
	ax & + & y & + & 2z & = & \alpha\\
	2x & - & y & - & z  & = & \gamma}$}}
\hfill
\begin{minipage}{.65\textwidth}
	\begin{lstlisting}[linewidth=\textwidth, language={[LaTeX]TeX}]
	$\systa{rororor}{%
	x  & + &ay & - & z  & = & \beta \\
	ax & + & y & + & 2z & = & \alpha\\
	2x & - & y & - & z  & = & \gamma}$
	\end{lstlisting}
\end{minipage}

\setlength{\leftskip}{0pt}
\setlength{\textwidth}{18cm}%


\end{document}
%%---------------------------------------------------------------%%
%                                                                 %
%                        PACKAGE IMPORTS                          %
%                        Module : Common                          %
%                                                                 %
%        All packages should be compatible with XeLaTeX           %
%                                                                 %
%%---------------------------------------------------------------%%


% Global libraries
\usepackage[textwidth=18cm,bottom=2cm,top=2cm]{geometry} % pages borders
%\usepackage[top=2cm, bottom=2cm, outer=6cm, inner=1.5cm, marginparwidth=5cm, marginparsep=.75cm]{geometry}
\usepackage[french]{babel} % language type annotations

% Graphics
\usepackage{graphicx} % improve include graphics
\usepackage{tikz} % for drawing
\usepackage[most, breakable]{tcolorbox} % to have fill paterns for TikZ
\usepackage{tikzscale} % in order to scale TikZ pictures

% Tabular
\usepackage{booktabs} % to have predefined tab rules
\usepackage{array} % to define new column types
\usepackage{longtable} % in order to have multipage tables

% Colors
%\usepackage{color}
\usepackage{xcolor} % to have access to every immaginable color

% TikZ libraries
\usetikzlibrary{fit} 
\usetikzlibrary{graphs} % package for graphics representation
\usetikzlibrary{positioning} % possitionning inside TikZ picture
\usetikzlibrary{arrows} % beautiful arrows with TikZ
\usetikzlibrary{calc} % enable computation of values insidecture
\usetikzlibrary{decorations} % decoration for TikZ picture
\usetikzlibrary{matrix} % matrix of nodes
\usetikzlibrary{angles} % angle helpers

% Characters
\usepackage{pifont} % in order to have checkmarks and crosses
\usepackage{eurosym} % to have euro symbol

% TOC and titles
\usepackage{caption} % to redefine caption styles
\usepackage[explicit,pagestyles]{titlesec} % redefine titles style
\usepackage{titletoc} % for partial TOC in chapters

% LaTeXing
\usepackage{etoolbox} % to have ifstrempty and other command in macros
\usepackage[unicode, hidelinks]{hyperref} % math formulas in section names

% Headers and footers
\usepackage{fancyhdr} % for fancy headers and footers

% Multicolumn
\usepackage{multicol} % provide multicolumn environments

% Index
\usepackage{makeidx} % enable the computation of document index

% Mathematics
\usepackage[d]{esvect} % Fancy vector arrows
\usepackage{tkz-tab} % For variation arrays

% Itemize
\usepackage{enumitem} % Allow to set values and positions for items

% Testing
\usepackage{lipsum} % for testing
\tcbuselibrary{listings,skins}

% Fonts
%============ KURIER ==========================
%\usepackage[math]{kurier} % main font
%\usepackage[T1]{fontenc} % font encoding
%\DeclareTextCommandDefault{\nobreakspace}{\leavevmode\nobreak\ } % Because T1 disable this command, we redefine it
%============ EULER ===========================
%\usepackage[T1]{fontenc}
%\usepackage{mathpazo} % add possibly `sc` and `osf` options
%\usepackage[T1]{eulervm}
%============ MATH DESIGN =====================
\usepackage{fontawesome} % to have web fonts
\usepackage[bitstream-charter]{mathdesign} % nice math and common fonts






\iffalse
%% Libraries for graphics and colours
\usepackage{changepage} % for adjustwidth command
\usepackage{colortbl}

\usepackage{mdframed} % permet de couper du texte entre plusieurs pages
\usepackage{framed}


\usepackage{stmaryrd}
\usepackage{ifthen} % use to make if then conditions


\usepackage{capt-of}
\usepackage{xspace}

\usepackage[normalem]{ulem}
\usepackage{url}
\usepackage{subcaption}
\usepackage{remreset}
\usepackage[framemethod=tikz]{mdframed}

\usepackage{listings}

%% Other libraries
\usepackage{notoccite} % helps for quotations
\usepackage{float} % helps to place figures
\usepackage{setspace} % in order to help adding space in the document
\usepackage{multirow} % in order to use multirow in a table

\usepackage{rotating} % to have rotating boxes
\usepackage[makeroom]{cancel} % cancel symbols in maths
\usepackage{marvosym}
\usepackage{xfrac}
\fi
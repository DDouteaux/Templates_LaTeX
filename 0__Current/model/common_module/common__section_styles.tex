%%-------------------------------------------------------------------%%
%                                                                     %
%           STYLE COLLECTIONS FOR SECTION, CHAPTER...NAMES            %
%                                                                     %
%%-------------------------------------------------------------------%%


% Basic style to fully customize section name, figure and separator.
% No other fioriture than given configuration will be used.
%
% Parameters :
%	- Chapter number
%	- Chapter name
%
% Name for configuration : basic
\newcommand\sectionBasic[2]{\textbf{#1}~~\textbf{#2}}


% Style with big figure and section name pushed to the right.
%
% Parameters :
%	- Chapter number
%	- Chapter name
%
% Name for configuration : numberInSquare
\newcommand\sectionNumberInSquare[2]{%
	\hspace*{\sectionLeftSpace}\textcolor{\sectionNumberColor}{\fbox{\textbf{#1}}}~~\textcolor{\sectionNameColor}{\textbf{#2}}
}


% Chapter style with big figure and section name after.
% The whole is underlined by a thin line.
%
% Parameters :
%	- Chapter number
%	- Chapter name
%
% Name for configuration : greatNumber
\newcommand\sectionGreatNumber[2]{%
  \noindent%
  \fontsize{30}{1}\selectfont%
  \textit{\Huge{%
    \textbf{#1.~~}%
  }}
  \fontsize{15}{1}\selectfont%
  \textit{\textbf{#2}}\par%
}


% Chapter style with big figure and chapter name pushed to the right.
% The whole is underlined by a thin line.
%
% Parameters :
%	- Chapter number
%	- Chapter name
%
% Name for configuration : numberInMargin
\newcommand\sectionNumberInMargin[2]{
	\begin{changemargin}{-1cm}{0cm}\textbf{#1}~~\textbf{#2}\end{changemargin}
	\vspace*{-.25cm}
}


% Chapter style with big figure and chapter name pushed to the right.
% The whole is underlined by a thin line.
%
% Parameters :
%	- Chapter number
%	- Chapter name
%
% Name for configuration : bigFigureAndRule
\newcommand\sectionBigFigureAndRule[2]{
	\begin{flushright}
		\textbf{\fontsize{70}{60}\selectfont #1}
		\vspace*{1cm}

		\textbf{\fontsize{30}{40}\selectfont #2}

		\rule{13cm}{1pt}
	\end{flushright}
	\vspace*{1.5cm}
}


% Chapter style with big figure and chapter name pushed to the right.
% Both are separated with a vertical line.
%
% TODO: Finish this style
%
% Parameters :
%	- Chapter number
%	- Chapter name
%
% Name for configuration : bigFigureAndVerticalRule
\newcommand\sectionBigFigureAndVerticalRule[2]{
	\begin{flushright}
		\textbf{\fontsize{70}{60}\selectfont #1}~~~~\rule[-1cm]{3pt}{3cm}~~~~\textbf{\fontsize{30}{40}\selectfont #2}

		\rule{13cm}{1pt}
	\end{flushright}
}


% Similar to chapterBigFigureAndVerticalRule but with a uniform currentColor background.
% Moreover, chapter number and names are alignes on the right.
%
% TODO: Finish this style
%
% Parameters :
%	- Chapter number
%	- Chapter name
%
% Name for configuration : bigFigureAndBackground
\newcommand\sectionBigFigureColoredBackground[2]{
	\begin{changemargin}{-1.5cm}{-1.5cm}
		\noindent\textcolor{-red!75!green!50}{\rule{\paperwidth}{8cm}}\hspace*{-3cm}\textcolor{white}{\rule{1.5pt}{7cm}}\newline
	\end{changemargin}
}


% Chapter style with background made with an image. The chapter number and name
% are located centered beneath the image, inside a cartridge.
%
% Parameters :
%	- Chapter number
%	- Chapter name
%
% Name for configuration : backgroundImage
\newcommand\sectionBackgroundImage[2]{
	\vspace*{-2.8cm}%
	\begin{changemargin}{-1.5cm}{-1.5cm}
		\noindent\begin{tikzpicture}%
		  \pgfdeclarelayer{background}%
		  \pgfdeclarelayer{foreground}%
		  \pgfsetlayers{background,foreground}%
		  \begin{pgfonlayer}{foreground}%
			\node (separator) [color=blue, fill=blue, draw, rectangle, minimum width=21cm, inner sep=1pt] at (-10.5,0) {};%
			\node (section_name) [fill=white, font=\fontsize{25}{1}\selectfont, inner sep=.3cm, text depth=.35ex] at (separator.center) {\textcolor{blue}{#1~~\raisebox{2pt}{$\bullet$}}~~\textbf{#2}\hspace*{.2cm}};%
			\fill [white] ([xshift=-2pt]section_name.north east) -- (section_name.north east) -- ([xshift=.5cm]section_name.east) -- (section_name.south east) --([xshift=-2pt]section_name.south east) -- cycle ;%
			\fill [white] ([xshift=2pt]section_name.north west) -- (section_name.north west) -- ([xshift=-.5cm]section_name.west) -- (section_name.south west)  --([xshift=2pt]section_name.south west) -- cycle ;%
			\draw [color=blue, line width=2pt] (section_name.north east) -- ([xshift=.5cm]section_name.east) -- (section_name.south east) -- (section_name.south west) -- ([xshift=-.5cm]section_name.west) -- (section_name.north west) -- cycle;%
		  \end{pgfonlayer}%
		  \begin{pgfonlayer}{background}%
			\path [fill tile image*={width=\paperwidth}{images/default4.JPG}] (-21,10) rectangle (0,0);%
		  \end{pgfonlayer}%
		\end{tikzpicture}%
	\end{changemargin}
}


% Chapter style with background made with an image. The chapter number and name
% are located top left in a transparent full text width line. It also auto include
% a partial TOC in a cartridge in the right.
%
% Parameters :
%	- Chapter number
%	- Chapter name
%
% Name for configuration : backgroundImageTransparencyWithTOC
\newcommand\sectionBackgroundImageTransparency[2]{
	\vspace*{-2.8cm}%
	\begin{changemargin}{-1.5cm}{-1.5cm}
		\noindent\begin{tikzpicture}%
		  \pgfdeclarelayer{background}%
		  \pgfdeclarelayer{foreground}%
		  \pgfsetlayers{background,foreground}%
		  \begin{pgfonlayer}{foreground}%
			\node (separator) [color=white, fill=blue, opacity=0, draw, rectangle, minimum width=21cm, inner sep=1pt] at (-10.5,0) {};%
			\node (section_name) [fill=white, opacity=.55, text opacity=1, text width=\paperwidth, anchor=west, font=\fontsize{25}{1}\selectfont, inner sep=.4cm, text depth=.35ex] at (-21,8.25) {\thColor{\textbf{#1. #2}}};%
			\node (section_toc)  [rounded corners, fill=white, draw=blue, fill opacity=.55, text opacity=1, minimum height=7cm, text width=.55\paperwidth, anchor=north west, inner sep=.2cm] at (-12.25,6.25) {\normalsize\begin{minipage}{\linewidth}\startcontents\printcontents{l}{1}{\setcounter{tocdepth}{1}}\end{minipage}};%
		  \end{pgfonlayer}%
		  \begin{pgfonlayer}{background}%
			\path [fill tile image*={width=\paperwidth}{images/default4.JPG}] (-21,10) rectangle (0,0);%
		  \end{pgfonlayer}%
		\end{tikzpicture}%
	\end{changemargin}
	\vskip .5cm
}


% Chapter style with background made with an image. The chapter number and name
% are located top left in a transparent full text width line. Partial TOC is not
% auto-included with this shapre
%
% Parameters :
%	- Chapter number
%	- Chapter name
%
% Name for configuration : backgroundImageTransparencyNoTOC
\newcommand\sectionBackgroundImageTransparencyWithoutToc[2]{
	\vspace*{-2.8cm}%
	\begin{changemargin}{-1.5cm}{-1.5cm}
		\noindent\begin{tikzpicture}%
		  \pgfdeclarelayer{background}%
		  \pgfdeclarelayer{foreground}%
		  \pgfsetlayers{background,foreground}%
		  \begin{pgfonlayer}{foreground}%
			\node (separator) [color=white, fill=blue, opacity=0, draw, rectangle, minimum width=21cm, inner sep=1pt] at (-10.5,0) {};%
			\node (section_name) [fill=white, opacity=.55, text opacity=1, text width=\paperwidth, anchor=west, font=\fontsize{25}{1}\selectfont, inner sep=.4cm, text depth=.35ex] at (-21,8.25) {\thColor{\textbf{#1. #2}}};%
		  \end{pgfonlayer}%
		  \begin{pgfonlayer}{background}%
			\path [fill tile image*={width=\paperwidth}{images/default4.JPG}] (-21,10) rectangle (0,0);%
		  \end{pgfonlayer}%
		\end{tikzpicture}%
	\end{changemargin}
}
\newcommand\contextext{L'histoire du calcul infinitésimal remonte à l'Antiquité. Sa création est liée à une polémique entre deux mathématiciens : Isaac Newton et Gottfried Wilhelm Leibniz. Néanmoins, on retrouve chez des mathématiciens plus anciens les prémices de ce type de calcul : Archimède, Pierre de Fermat et Isaac Barrow notamment.
\newline
La notion de nombre dérivé a vu le jour au xviie siècle dans les écrits de Gottfried Wilhelm Leibniz et d'Isaac Newton qui le nomme fluxion et qui le définit comme « le quotient ultime de deux accroissements évanescents ».
\newline
Le domaine mathématique de l'analyse numérique connut dans la seconde moitié du xviie siècle une avancée prodigieuse grâce aux travaux de Newton et de Leibniz en matière de calcul différentiel et intégral, traitant notamment de la notion d'infiniment petit et de son rapport avec les sommes dites intégrales.}

% Disable page numbering
%\pagenumbering{gobble}

\newcommand\textbi[1]{\textbf{\textit{#1}}}
%% New headers and footers
\pagestyle{fancy}
\fancyhead[RE,LO]{}
\fancyhead[RO]{\includegraphics[height=0.55cm]{images/Logo/\headerLogo}~~\thColor{$\bullet$}~~\thColor{Deuxième version du template}}
\fancyhead[LE]{\thColor{Deuxième version du template}~~\thColor{$\bullet$}~~\includegraphics[height=0.55cm]{images/Logo/\headerLogo}}
\fancyfoot[RE,LO]{}
\fancyfoot[CE,CO]{}
\fancyfoot[RO]{\thColor{\thepage~~$\bullet$~~\thColor{Année 2017}}}
\fancyfoot[LE]{\thColor{\thColor{Année 2017}~~$\bullet$~~\thepage}}
\renewcommand{\headrulewidth}{0pt}
\renewcommand{\footrulewidth}{0pt}
\fancyhfoffset[OR, EL]{\marginparsep + \marginparwidth}

%% Configurations de base pour les liens hyperref
\hypersetup{hidelinks,
		pdfauthor={Damien Douteaux}, 
	        pdftitle={Projet de recherche et développement},
		pdfkeywords={Centrale Lyon, IRD, Projet}}

%% Change margins environment
\def\changemargin#1#2{\list{}{\rightmargin#2\leftmargin#1}\item[]}
\let\endchangemargin=\endlist 

%% Sections needing completion ands type of completions expected
\newcommand{\needCompletion}[1]{\newline\textcolor{purple}{\textit{#1}}\newline}

%% Emphasing elements in a table
\newcommand{\emphasizeTab}{\textcolor{bordeau}{\up{$\circledast$}}}

%% Glossary entry
\newcommand{\itemGlossaire}[3]{\par\textcolor{themeColor}{\textbf{#1}}\label{#2}\hspace{0.7cm}{#3}\vspace{0.2cm}}
\newcommand{\itemAcro}[4]{\par\textcolor{themeColor}{\textbf{#1} (#2)}\label{#3}\hspace{0.7cm}{#4}\vspace{0.2cm}}

%% Other unused commands
\newcommand\roundboxwithline[2]{\noindent\tikz{[line width=1]
\coordinate (leftshift) at (0,0);
\node (sectionname) [draw,rectangle,inner sep = 5,outer sep = 0,rounded corners = 1mm,font=\fontsize{#1}{1}\it\selectfont,color=themeColor,text opacity = 1,right] at (1,0) {#2};
\draw [color=themeColor] (leftshift) -- (sectionname.west);
\draw [color=themeColor] (sectionname.east) -- (\linewidth-1,0);}} 

%% Table interline value
\renewcommand*\arraystretch{1.5}

%% Don't forget me mark
\newcommand\beware{\up{\textcolor{bordeau}{(!)}}}

%% Rectif indices
\newcommand{\rectif}{\hspace{0.05cm}}

%% Special include when newpage is needed
\newcommand{\inputn}[1]{\newpage\input{#1}}

%% Mise en avant de résultats intermédiaires
\newcommand{\iResult}[2]{\iiResult{Intermediate outcome :}{#1}{passengers/#2}{!25}}
\newcommand{\itResult}[2]{\iiResult{Intermediate outcome :}{#1}{trips/#2}{!25}}

\newcommand{\isResult}[3]{\newline\vskip\hspace*{\fill}\large{\textcolor{themeColor}{#1} $#2$ #3}\hspace*{\fill}\newline}

\newcommand{\iiResult}[4]{\newline\vskip\noindent\hspace*{\fill}\colorbox{themeColor}{\rule{0cm}{1cm}\hspace{.07cm}}\colorbox{themeColor#4}{\hspace{.1cm}\rule{0cm}{1cm}\raisebox{.3cm}{\large{\textcolor{themeColor}{#1} $#2$ #3\hspace{.1cm}}}}\hspace*{\fill}}

\newcommand{\qpays}{$\mathcal{Q}_\textnormal{pays}$}
\newcommand{\qvolprod}{$\mathcal{Q}_\textnormal{volume production}$}
\newcommand{\qprofil}{$\mathcal{Q}_\textnormal{profil}$}
\newcommand{\qinteret}{$\mathcal{Q}_\textnormal{aspect important}$}
\newcommand{\qpayback}{$\mathcal{Q}_\textnormal{payback}$}
\newcommand{\qbudget}{$\mathcal{Q}_\textnormal{budget}$}
\newcommand{\qagriculture}{$\mathcal{Q}_\textnormal{agriculture}$}

\setlength{\multicolsep}{6.0pt plus 2.0pt minus 1.5pt}


\newcommand{\getmin}[2]{
  \ifdim#1pt>#2pt%
  #1%
  \else%
  #2%
  \fi%
}
%%% Local Variables:
%%% mode: latex
%%% TeX-master: "../TemplateV2"
%%% End:

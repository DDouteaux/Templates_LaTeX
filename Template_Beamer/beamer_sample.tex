\documentclass{beamer}
\usepackage[utf8]{inputenc}
\usepackage[T1]{fontenc}
\title{There Is No Largest Prime Number}
\date[ISPN ’80]{27th International Symposium of Prime Numbers}
\author[Euclid]{Euclid of Alexandria \texttt{euclid@alexandria.edu}}
\date{Jeudi 16 février 2017}

\usetheme{texsx}

\begin{document}

\section{Début}
\begin{frame}[plain]
\titlepage
\end{frame}

\section{Milieu}

\begin{frame} 
	\frametitle{There Is No Largest Prime Number} 
	\framesubtitle{The proof uses \textit{reductio ad absurdum}.} 
	
	\begin{theorem}
		There is no largest prime number. 
	\end{theorem} 
	
	\begin{enumerate} 
		\item<1-| alert@1> Suppose $p$ were the largest prime number. 
		\item<2-> Let $q$ be the product of the first $p$ numbers. 
		\item<3-> Then $q+1$ is not divisible by any of them. 
		\item<1-> But $q + 1$ is greater than $1$, thus divisible by some prime
			number not in the first $p$ numbers.
	\end{enumerate}
\end{frame}

\section{Suite}
\begin{frame}{A longer title}
	\begin{itemize}
		\item one
		\item two
	\end{itemize}
\end{frame}

\section{Plan}
\begin{frame}
	\frametitle{Plan de la présentation}
	\insertverticalnavigation{3cm}
\end{frame}

\end{document}

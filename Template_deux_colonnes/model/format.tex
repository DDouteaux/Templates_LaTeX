%
% Applique un formatage sur le texte fourni
% Params :
%     1 - Le format demandé
%     2 - Le texte à mettre en forme
%
\makeatletter
\newcommand{\applyFormat}[2]{%
  \ifnum\pdf@strcmp{#1}{textbi}=0%
  \textbi{#2}%
  \else\ifnum\pdf@strcmp{#1}{textbf}=0%
  \textbf{#2}%
  \else\ifnum\pdf@strcmp{#1}{textit}=0%
  \textit{#2}%
  \else\ifnum\pdf@strcmp{#1}{textsc}=0%
  \textsc{#2}%
  \else\ifnum\pdf@strcmp{#1}{textbic}=0%
  \textbic{#2}%
  \else\ifnum\pdf@strcmp{#1}{textbfc}=0%
  \textbfc{#2}%
  \else\ifnum\pdf@strcmp{#1}{textic}=0%
  \textic{#2}%
  \else\ifnum\pdf@strcmp{#1}{textscic}=0%
  \textbic{#2}%
  \else\ifnum\pdf@strcmp{#1}{textscc}=0%
  \textscc{#2}%
  \else\ifnum\pdf@strcmp{#1}{thColor}=0%
  \thColor{#2}%
  \fi\fi\fi\fi\fi\fi\fi\fi\fi\fi%
}
\makeatother

\newcommand\textbi[1]{%
  \textbf{\textit{#1}}%
}

\newcommand\textbic[1]{%
  \textbf{\textit{\thColor{#1}}}%
}

\newcommand\textbfc[1]{%
  \textbf{\thColor{#1}}%
}

\newcommand\textscic[1]{%
  \textsc{\textit{\thColor{#1}}}%
}

\newcommand\textscc[1]{%
  \textsc{\thColor{#1}}%
}

\newcommand\textic[1]{%
  \textit{\thColor{#1}}%
}
  

%%% Local Variables:
%%% mode: latex
%%% TeX-master: "../Template"
%%% End:

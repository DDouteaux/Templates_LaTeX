% Renew section like command

\newcommand\invisiblesection[1]{%
  \addcontentsline{toc}{section}{\protect#1}%
  \sectionmark{#1}
}

\frenchbsetup{IndentFirst=false}



%% New paragraph style (general code)
\newlength{\mytextsize}

\makeatletter
	\setlength{\mytextsize}{\f@size pt}
\makeatother

\newcommand\crule[3][themeColor]{\textcolor{#1}{\rule{#2}{#3}}}

\newenvironment{generalparagraph}[4]{
  \renewcommand{\currentColor}{#4}
  \noindent\hspace*{-.05cm}\raisebox{-.2cm}[0cm][.25cm]{
    \tikz{%
      \node (part_title) [draw, color=\headersColor, text depth=0.25ex, minimum width=3.1cm, align=left] at (0,0) {#2{{#1}}};%
      \fill [\headersColor] (part_title.north west) rectangle ([xshift=-.1cm]part_title.south west);%
    }%
  }%
  \hspace{1cm}%
}{
    \ignorespacesafterend\hspace*{\fill}\linebreak
    \renewcommand{\currentColor}{\headersColor}
}

\newenvironment{generalminiparagraph}[4]{
    \renewcommand{\currentColor}{#4}
    \par\vspace{0.3cm}\noindent\tikz{\draw [color=#4, fill=#4] (0,0) rectangle (\mytextsize/2-1,\mytextsize-2);
			} \textcolor{#4}{#2{{#1}}}
    \hspace{1cm}\noindent#3
}{
    \par\vspace{0.3cm}\ignorespacesafterend
    \renewcommand{\currentColor}{themeColor}
}

%% Several types of useful paragraphs (using previous source code)
% Avec une en-tête de paragraphe complète ||| || |
\newenvironment{remarque}{\begin{generalparagraph}{Remark}{\bfseries}{\normalfont}{vertforet}}{\end{generalparagraph}}
\newenvironment{attention}{\begin{generalparagraph}{Attention}{\bfseries}{\normalfont}{bordeau}}{\end{generalparagraph}}
\newenvironment{information}{\begin{generalparagraph}{Information}{\bfseries}{\normalfont}{orange}}{\end{generalparagraph}}
\newenvironment{paragraphe}[1]{\begin{generalparagraph}{#1}{\bfseries}{\normalfont}{themeColor}}{\end{generalparagraph}}
\newenvironment{question}[1]{\begin{generalparagraph}{Question #1}{\bfseries\large}{\normalfont}{themeColor}}{\end{generalparagraph}}
% Avec une en-tête de paragraphe réduite ||
\newenvironment{convention}{\begin{generalminiparagraph}{Convention}{\bfseries}{\normalfont}{vertforet}}{\end{generalminiparagraph}}
\newenvironment{lecture}{\begin{generalminiparagraph}{Lecture}{\bfseries}{\normalfont}{vertforet}}{\end{generalminiparagraph}}
\newenvironment{notation}{\begin{generalminiparagraph}{Notation}{\bfseries}{\normalfont}{vertforet}}{\end{generalminiparagraph}}
\newenvironment{demonstration}{\begin{generalminiparagraph}{Démonstration}{\bfseries}{\normalfont}{orange}}{\proved\end{generalminiparagraph}}
\newenvironment{exempleInterne}{\begin{generalminiparagraph}{Exemple}{\bfseries}{\normalfont}{themeColor}}{\end{generalminiparagraph}}
\newenvironment{exemple}[1]{\begin{exempleInterne}\textit{#1}\end{exempleInterne}}

%% Unumbered sections (Conclusion, Abstract, Bibliography)
\newcommand{\nnsection}[1]{\section*{#1}\addcontentsline{toc}{section}{#1}}
\newcommand{\appendixsection}{\newpage\invisiblesection{Annexes}\vspace*{\fill}
\hfill{\fontsize{75}{1}\selectfont\textbf{\thColor{Annexes}}}\hspace*{\fill}
\vspace{2cm}
\PartialToC
\vspace*{\fill}
\vspace*{\fill}
\newpage


%%% Local Variables:
%%% mode: latex
%%% TeX-master: "../../Rapport_dreches"
%%% End:
}

%% New subParagraph command
\newcommand{\subParagraphe}[1]{\par\vspace{0.3cm}\noindent\textcolor{\currentColor}{\textbf{\textit{#1}}}\hspace{.5cm}}

%% New subsubParagraph command
\newcommand{\subsubParagraphe}[1]{\par\vspace{0.2cm}\textbf{#1}\hspace{0.5cm}}

%% Pour ne pas avoir le "Références" qui s'affiche en créant la bibliographie
\makeatletter
\def\pasdetitreici{\def\section{\@ifstar\@gobble\@gobble}}
\makeatother



%%% Local Variables:
%%% mode: latex
%%% TeX-master: "../Rapport_dreches"
%%% End:
